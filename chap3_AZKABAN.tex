\chapter{Model-informed classification of broadband acoustic backscattering from zooplankton in an \textit{in situ} mesocosm}
\label{chap:modelled}



Muriel Dunn$^{1,2}$, Chelsey McGowan-Yallop$^3$ Geir Pedersen$^4$, Stig Falk-Petersen$^5$, Malin Daase$^6$, Kim Last$^3$, Tom J. Langbehn$^7$, Sophie Fielding$^8$, Andrew S. Brierley$^9$, Finlo Cottier$^{3,6}$, S\"{u}nnje L. Basedow$^6$, Lionel Camus$^1$, and Maxime Geoffroy$^{2,6}$\\

$^1$ Akvaplan-niva AS, Fram Centre, Postbox 6606, Stakkevollan, 9296 Tromsø, Norway \\
$^2$ Center for Fisheries Ecosystems Research, Fisheries and Marine Institute of Memorial University of Newfoundland and Labrador, St. John's, A1C 5R3, NL, Canada\\
$^3$ Scottish Association for Marine Science, Oban, Argyll PA37 1QA, UK\\
$^4$ Institute for Marine Research, 5005 Bergen, Norway\\
$^5$ Independent scientist, 9012 Tromsø\\
$^6$ Department of Arctic and Marine Biology, UiT The Arctic University of Norway, 9019 Tromsø, Norway\\
$^7$ Department of Biological Sciences, University of Bergen, 5020 Bergen, Norway
$^8$British Antarctic Survey, Natural Environment Research Council, High Cross, Cambridge CB30ET, United Kingdom \\
$^9$Pelagic Ecology Research Group, School of Biology, Scottish Oceans Institute, Gatty Marine Laboratory, University of St Andrews, St Andrews, KY16 9TS, United Kingdom \\


Submitted to \textit{ICES Journal of Marine Sciences} (14 August 2023) \\

\section{Abstract}
Classification of zooplankton to species with broadband echosounder data could increase the taxonomic resolution of acoustic surveys and reduce the dependence on net and trawl samples for ``ground truthing." Supervised classification with broadband echosounder data is limited by the acquisition of validated data required to train machine learning algorithms (`classifiers'). We tested the hypothesis that acoustic scattering models could be used to train classifiers for remote classification of zooplankton. Three classifiers were trained with data from scattering models of four Arctic zooplankton groups (copepods, euphausiids, chaetognaths, and hydrozoans). We evaluated classifier predictions against observations of a mixed zooplankton community in a submerged purpose-built mesocosm (12 m$^3$) insonified with broadband transmissions (185 to 255 kHz). The mesocosm was deployed from a wharf in Ny-Ålesund, Svalbard, during the Arctic polar night in January 2022. We detected 7,722 tracked single target detections which were used to evaluate the classifier predictions of measured zooplankton targets. The classifiers could differentiate copepods from the other groups reasonably well, but they could not differentiate euphausiids, chaetognaths, and hydrozoans reliably due to the similarities in their modelled target spectra. We recommend that model-informed classification of zooplankton from broadband acoustic signals be used with caution until a better understanding of in situ target spectra variability is gained.

\section{Introduction}
Acoustic target classification of zooplankton is needed to improve our understanding of variability in zooplankton spatio-temporal distribution and community composition. In the past decade, the commercial availability of broadband echosounders has made it possible to characterise the backscattering spectra of aquatic targets over a continuous frequency range \citep{Bassett2018}. Compared to conventional narrowband echosounder methods, the wider bandwidth of frequency-modulated (FM) echosounders offers the potential for improved classification of fish and zooplankton \citep{BenoitBird2020}. In addition, pulse-compression signal processing of broadband data improves the range resolution and the signal-to-noise ratio, enabling weak zooplankton targets to stand out above the stochastic background noise  \citep{Chu1998, Ehrenberg2000}.These improvements have made it possible to distinguish smaller and acoustically weaker individual targets, such as mesozooplankton (0.2 to 20 mm), offering the potential for target classification using the target strength (TS) - frequency response spectra (TS(f), hereafter `target spectra') as a predictive feature \citep{Bandara2022}. \\
Machine learning (ML), a field of artificial intelligence, is an increasingly popular tool for target classification in fisheries acoustics, reflecting a broader trend of AI applications in the marine sciences \citep{Beyan2020, Malde2020}. ML methods are objective, efficient, and can handle the large, complex datasets associated with broadband sampling \citep{Malde2020}. In short, supervised classification algorithms are trained to predict the class of new, unidentified samples with reference to scattering spectra from labelled training samples (i.e., samples for which the class is known) to optimise the classification function. In a fisheries acoustics context, the class is typically the species (or a broader functional group, e.g., based on gross anatomical properties) of the target or aggregation. The feature variables used to predict the class of each target may include various acoustic features (e.g., backscattering strength and derived quantities), often in combination with geometric features (e.g., school length and height; \citealt{Proud2020}) or bathymetric features (e.g., distance from the seabed)  \citep{Korneliussen2018}. Machine learning algorithms improve the potential for real-time target classification and subsequent analysis (such as density estimates; \citealt{Blackwell2020}) for the increasing use of autonomous or remotely operated vehicles equipped with echosounders (e.g., \citealt{Ludvigsen2018, DeRobertis2019, Malde2020, Dunn2022}). However, a significant obstacle to applying supervised classification in fisheries acoustics is the collection of labelled observations to train the algorithms \citep{Handegard2021}. \\
Labelled observations of TS and target spectra have been measured using various direct sampling or remote sensing methods, all of which have limitations. For example, directed trawl sampling of acoustic targets in areas with high densities of the species of interest has been used for jellyfish \citep{Brierley2001}, Antarctic krill \citep{Hewitt1996} and mesopelagic fish \citep{Sobradillo2019} but this method prone to sampling biases like net avoidance and acoustic shadowing of weaker targets \citep{Pena2018}. Optical verification has been used to validate acoustic targets, for example, krill \citep{Lawson2006} and salps \citep{Wiebe2010}, but has limited range resolution, especially for small targets \citep{Trenkel2011} and is further limited by avoidance of the external light source \citep{Geoffroy2021}. Controlled tank experiments 
 with zooplankton (e.g., \citealt{Pauly1998, Stanton1998, McGehee1998}) have typically relied on purpose-built or laboratory sonars \citep{Amakasu2006, Conti2005} because there are physical limits associated with (large and powerful) commercially available echosounders (i.e., beam angle and near-field range; \cite{Simmonds2008}). Controlled cage experiments have been used to measure the acoustic signal of large Antarctic krill (e.g., \citealt{Foote1990}), jellyfish \citep{Monger1998} and fish (e.g., \citealt{Gugele2021, Legua2017}), but measurements of mesozooplankton remain challenging because detection of weak scatterers requires a cage designed to minimise noise and reverberation \citep{Knutsen1997}.\\
Model-informed classification theoretically removes the need to collect measurements of known targets for use as labelled training data (e.g., \cite{Cotter2021}. Validated scattering models (e.g., \citealt{Korneliussen2003, Cotter2021, Pena2018}) provide theoretical frequency response spectra for each class (e.g., species) expected to be present in the acoustic data.  Sound scattering models are considered validated when predictions of acoustic backscatter are comparable to benchmark models \citep{Gastauer2019}. Benchmark models are predictions of acoustic backscatter from exact or approximate analytical models and serve to find the limitations and validity domain of sound scattering models \citep{Jech2015}. These modelled spectra are then used as labelled training data for machine learning classification algorithms (hereafter, `classifiers'). This approach has been used to classify scatterers into gross anatomical groups based on their acoustic properties for mesopelagic species \citep{Cotter2021} and reef fish \citep{Roa2022}. However, to our knowledge, model-informed classification of target spectra has not yet been validated for any species.  \\
This study aims to evaluate the validity and reliability of model-informed classification for the target spectra of zooplankton species with similar gross anatomical properties and size distributions. We applied model-informed classification to a mixed assemblage of Arctic mesozooplankton that was dominated by fluid-like species, i.e., animals with sound scattering properties similar to water (e.g., euphausiids, copepods, and salps) \citep{Stanton2000b}. The objectives were threefold: (1) to design an in situ mesocosm experiment to insonify zooplankton in a near-natural environment with minimal background noise and reverberation; (2) to evaluate the performance of classifiers trained with scattering models for differentiating weakly backscattering mesozooplankton groups; (3) to validate the classifier predictions on a known community of zooplankton. We conclude by providing recommendations for model-informed classification of target spectra.

\section{Material and methods}
\subsection{Study area and zooplankton collection}
Zooplankton were collected in Kongsfjorden, Svalbard, from the R/V \textit{Helmer Hanssen} using a Tucker trawl (1 m$^2$ opening and 1000 µm mesh size, 10 minutes at 3 m s$^{-1}$)) on the night of 15 January 2022 (Figure~\ref{fig:FigureAZKABAN1}). Twelve Tucker trawl tows were taken at the depth of the strongest sound scattering layer ($\sim$150 m) as seen from the vessel's echosounder (Kongsberg Maritime AS, Horten, Norway; Simrad EK60, 18 and 38 kHz, 1.024 ms pulse duration, 2 Hz ping rate). Samples from all tows were combined and kept alive for up to 15 hours in running seawater and delivered unsorted to the wharf in Ny-Ålesund on 16 January. The zooplankton samples were stored overnight in three 100 L holding tanks with a low-pressure flow system of filtered ambient seawater ($\sim$2$^{\circ}$C) at the Kings Bay Marine Laboratory. An additional Tucker trawl sample collected on 15 January was preserved in 4\% formaldehyde-in-seawater solution buffered with hexamine and stored for species shape analysis.\\

\munepsfig[scale=.95]{FigureAZKABAN1}{Study area in Kongsfjorden with locations of the mesocosm experiment from the wharf in Ny-Ålesund (red square) and Tucker trawl deployments for the experiment (blue circles with some overlap; n=12). The yellow circle indicates the Tucker trawl deployment from which zooplankton was preserved for morphometric analyses (yellow circle; n=1). The red box in the inset shows the location of the study area within the Svalbard archipelago.}


\subsection{Mesocosm design and experiment}
Acoustic data were collected on 17 January 2022 using a mesocosm deployed from a wharf in Ny-Ålesund (Figure~\ref{fig:FigureAZKABAN1}). The mesocosm, or AZKABAN (Arrested Zooplankton Kept Alive for Broadband Acoustics Net experiment), was formed by a cuboid zooplankton net (3 m high, 2 m wide and 2 m long) with a 500 µm-mesh holding a volume of 12 m$^3$ (Figure~\ref{fig:FigureAZKABAN1}a). The net was mounted on the top section of an 8 m high by 2 m wide and 2 m long aluminium frame oriented vertically (Figure~\ref{fig:FigureAZKABAN1}a). Ropes attached eyelets on the net to the frame at each corner and along the edges.\\  

\munepsfig[scale=.6]{FigureAZKABAN2}{A) Schematic of the AZKABAN mesocosm with the small zooplankton net (left) and large fish net (left). Only the configuration with a small net (left) was used for this study to limit the volume of insonified mesozooplankton. The acoustic transceiver (yellow cylinder) is attached to the frame and the transducer (orange cylinder). There is a hole at the top of the net for the transducer face to be unobstructed inside the net. B) The AZKABAN mesocosm lifted with the crane at the end of the experiment.}

A 200 kHz nominal frequency transducer (ES200-7CDK-Split; Kongsberg Maritime AS) was mounted on a plate centred inside the mesocosm through a hole on the top panel of the net with the acoustic axis pointing directly down. A Wideband Autonomous Transceiver (WBAT; Kongsberg Maritime AS) was fastened to the frame to operate the transducer (Figure~\ref{fig:FigureAZKABAN2}). The AZKABAN frame was purpose-built by Havbruksstasjonen (Ringvassøya, Norway) and I designed the frame to contain the entire main lobe of a 7° opening beam angle transducer inside the net.\\
The AZKABAN mesocosm was deployed by crane and lowered into the sea (Figure~\ref{fig:FigureAZKABAN2}b). Zippers on the top and bottom panels of the net were used to add the alive and active species from the holding tanks into the submerged net. The frame was lowered such that the depth of the transducer face was approximately 0.5 m below the surface for the duration of the experiment. The mesocosm was recovered after three hours of data collection (Appendix~\ref{apdx:SuppMat2} Figure S1). The zooplankton were rinsed off the net and collected for species composition analysis. The species composition of the recovered mesocosm sample was analysed by identifying and counting 10\% of the total sample for all species with more than 1000 individuals. All other species were counted for the entire sample.\\
The mesocosm experiment was conducted on an unsorted assemblage to maintain a high detection probability (i.e., with large numbers of target animals in the enclosure). The sampling effort required to obtain sufficient animals for single-species experiments was deemed too great in time and hence expense. In addition, separating the live mesozooplankton from a mixed assemblage (as caught) into single species groups would have risked injuring or killing individuals. Using the unsorted mixed population meant that individual animals were handled minimally and that stress to them was minimised: this left it likely that natural swimming behaviour was preserved.

\subsection{Acoustic data collection and calibration}
During the AZKABAN experiment, the WBAT was programmed to transmit frequency-modulated pulses covering the entire available bandwidth from 185 to 255 kHz. The transmitted pulses had fast ramping, a pulse duration of 512 μs with 75 W transmit power, and a ping interval of 0.35 s. Simultaneous pinging of two split-beam transducers is not possible with a WBAT, so we had to restrict the bandwidth to that achievable by one transducer alone for the experiment. The simultaneous pinging of two or more transducers would improve the classification potential of broadband signals \citep{BenoitBird2020}. Of the available transducers with 7° beam width (120, 200 and 333 kHz), the 200 kHz transducer was chosen to have the greatest signal-to-noise ratio of the targeted species (mesozooplankton) while achieving a small wavelength to detect smaller zooplankton (7 mm; \citealt{Simmonds2008}. We used a short pulse length to resolve targets near the net boundary and reduce reverberation volume \citep{Soule1997}. \\
The acoustic system was calibrated on 19 January 2022 with two spheres made of tungsten carbide (WC) with 6\% cobalt binder and diameters of 38.1 mm and 22 mm \citep{Demer2015}. Calibrations were processed with the EK80 software (version 21.15; Kongsberg Maritime AS). The calibration parameters were calculated for each sphere (Appendix~\ref{apdx:SuppMat2} Figure S2) and combined.


\subsection{Scattering models}
The training dataset for the classification was created with scattering models for the most abundant taxonomic groups in the Tucker trawl samples ($\leq$ 1000 individuals). The most abundant were calanoid copepods, euphausiids, chaetognaths, and hydrozoans. All these groups are considered fluid-like scatterers with sound speed contrast (\textit{h}) and density contrast (\textit{g}) of 1 $\pm$ 5\% \citep{Stanton2000b}. Near-unity sound speed and density contrasts imply that the material properties of the scatterers are not significantly different from the surrounding medium (seawater). To model the scattering of the zooplankton groups, we chose the phase-compensated distorted wave Born approximation (PC-DWBA) model because the parameters of this model are flexible to geometry, material properties, and acoustic frequency ranges, which makes the model adequate for the broad range of fluid-like zooplankton groups in this study \citep{Chu1999, Gastauer2019}. The DWBA has been extensively tested \citep{Lavery2007}, and PC-DWBA model has been used to infer length or material properties for Antarctic krill, \textit{Euphausia superba} \citep{Amakasu2017}, decapod shrimp, \textit{Palaemonetes vulgaris} \citep{Chu2000}, and eggs of North Atlantic cod, \textit{Gadus morhua} \citep{Chu2003} by comparison of model outputs with measurements of known species (in controlled laboratory experiments, or concurrent trawl sampling). We ran 1000 model simulations for each zooplankton group using the ZooScatR package (version 0.5, \citealt{Gastauer2019}) with R (version 4.1.2) with shape, size, and material properties parameters chosen from distributions selected based on the basis of the mesocosm-experiment samples, the preserved sample or literature (Table~\ref{tab:modelparams2}). The modelled spectra were calculated with a 0.5 kHz frequency resolution.

\begin{muntab}{|l|l|l|l|l|}{modelparams2}{Scattering model parameters distributions for each zooplankton group. The distributions are log-normal: $L$(meanlog, sdlog), normal: $N$(mean, sd) and gamma: $\Gamma$(shape, rate), where sd is the standard deviation.}
\hline
Parameters & Copepods & Euphausiids & Chaetognaths & Hydrozoans \\
\hline
Modelled & \textit{Calanus } & \textit{Thysanoessa } & \textit{Parasagitta } & \textit{Aglantha } \\
species & \textit{glacialis} & \textit{inermis} & \textit{elegans} & \textit{digitale} \\
\hline
Length (mm) & $N$(3.3, 0.7)$^a$ & $L$(2.4, 0.3)$^d$ & $\Gamma$(10.6, 0.6)$^a$ & $L$(2.4,0.4)$^a$ \\
\hline
Length-to- & $N$(5.3, 0.9)$^a$ & $N$(11, 2)$^a$ & $N$(26, 8)$^a$ & $N$(2.8,0.5)$^a$ \\
width ratio & & & & \\
\hline
Density & $N$(0.997, 0.005)$^b$ & $N$(1.037, 0.005)$^b$ & $N$(1.030, 0.005)$^e$ & $N$(1.007, 0.005)$^f$ \\
contrast (\textit{g}) & & & & \\
\hline
Sound speed  & $N$(1.027, 0.007)$^b$ & $N$(1.026, 0.005)$^b$ & $N$(1.030, 0.005)$^e$ & $N$(1.007, 0.005)$^f$ \\
contrast (\textit{h}) & & & & \\
\hline
Orientation ($^\circ$) & $N$(90, 30)$^c$ & $N$(20, 20)$^e$ & $N$(0, 30)$^e$ & $N$(90, 30)$^g$ \\
\hline
\end{muntab}
\begin{flushleft}
$^a$ Measurements from the preserved sample with the distribution assessed as the best fit based on a 1:1 line between theoretical and empirical quantile in Q-Q plots.\\
$^b$ \citet{Kogeler1987}; February-March measurements.\\
$^c$ \citet{Blanluet2019}\\
$^d$ Measurements from a subsample of the mesocosm experimental sample. The distribution was assessed as the best fit based on a 1:1 line between theoretical and empirical quantiles in Q-Q plots.\\
$^e$ \citet{Lavery2007}\\
$^f$ Inferred from a comparison of measurements of hydrozoans from \citet{Monger1998}, \citet{Brierley2001} and \citet{Brierley2004} to model predictions.\\
$^g$ \citet{Monger1998} from swimming shape analysis.\\
\end{flushleft}

The preserved Tucker Trawl sample was diluted and subsampled on 22 June 2022 for imaging of copepods (n=70), euphausiids (n=20), chaetognaths (n=70), and hydrozoans (n=70). Images were taken with a Leica M205 C stereomicroscope fitted with a Leica MC170 HD camera, and shape analysis was performed with an image processing software, ImageJ (version 1.53, National Institutes of Health, USA). The shapes were processed with ZooScatR to calculate the length and length-to-width ratio. Large individuals (>16 mm) were measured with a ruler. For the euphausiids, the length distribution was calculated from a subsample of 77 individuals from the mesocosm experiment sample. The processed images were used to create a shape input for each zooplankton group and its scattering model (Appendix~\ref{apdx:SuppMat2} Figure S3).\\
Material properties of copepods vary in the literature \citep{Sakinan2019}. We selected \textit{g} and \textit{h} from \citet{Kogeler1987} because of the availability of measurements from the winter season (February-March) and the proximity of their measurements of \textit{Calanus} spp. to the Arctic, hereby Arctic copepods. For hydrozoans, literature values for density and sound speed contrast were limited; therefore, we inferred the values for \textit{g} and \textit{h} from a comparison of the measurements from \citet{Brierley2001}, \citet{Brierley2004} and \citet{Monger1998} to the model predictions, a method used by \citet{Lavery2007}.\\

\subsection{Acoustic data processing}
All acoustic data were processed in Echoview 13.0 (Echoview Software Pty Ltd, Hobart, Tasmania). Data analysis was restricted to the 1.0 to 2.25 m range to exclude the near-field region \citep{Simmonds2008} and the echo from the bottom of the net. The ``Single Target Detection - wideband" operator was applied to the pulse-compressed wideband data (Appendix~\ref{apdx:SuppMat2} Table S1). The minimum value for the compensated TS threshold was set to the minimum allowable value, -120 dB re 1 m$^2$, to allow for the detection of the weaker scatterers. The identified single targets were grouped into tracks using the ``Detect Fish Tracks" algorithm. We used conservative parameters to increase the likelihood of each track containing targets from one individual (Appendix~\ref{apdx:SuppMat2} Table S2). Tracks were visually assessed to remove outlier targets to further ensure that each track originated from only a single zooplankton target.\\
The target spectra of all single targets assigned to a track were exported from Echoview for analysis. All target spectra were calculated using a Fourier transform window size of 0.33 times the pulse length (0.25 m) with a 0.5 kHz resolution. The Fourier transform window size was selected as a compromise to maximise frequency resolution while minimising the likelihood of incorporating backscattering from multiple targets \citep{BenoitBird2020}.\\ 

\subsection{Noise level}
The noise level inside the mesocosm affected the minimum backscatter detectable from organisms. In this case, noise level is considered all unwanted signals, including background noise and reverberation from the cage. The noise level within AZKABAN was calculated using a 1-minute segment of data collected during a period of low single echo detections (11:25-11:26 UTC). First, single target detection was applied to the pulse-compressed TS with less stringent detection thresholds (Appendix~\ref{apdx:SuppMat2} Table S2) to identify all possible targets. Second, targets were removed from the dataset using a mask. The target masks covered entire pings to avoid contamination by side lobes associated with pulse compression from targets. The remaining signal was designated as noise. Finally, the noise level was calculated by exporting the median target strength frequency response profile for increments of 0.1 m depth bins.\\
Thereafter, when selecting single targets for the spectra analysis, targets were flagged (i.e., excluded from the analysis) if their target strength at nominal frequency (200 kHz) had a signal-to-noise ratio (SNR) of less than 10 dB \citep{Simmonds2008} when compared to the noise level at nominal frequency at the range of the target. We calculated the proportion of flagged targets below the SNR threshold relative to the total amount of targets. The full spectrum was not assessed because of the peaks and nulls in the target spectra.\\


\subsection{Classifier training}
Various algorithms have been used for acoustic target classification in previous fisheries acoustics studies, including k-Nearest Neighbours \citep{Cotter2021}, decision trees \citep{Fernandes2009, DElia2014}, random forests (e.g., \citealt{Gugele2021, Proud2020}), gradient boosting \citep{EscobarFlores2019}, support vector machines \citep{Roberts2011, Roa2022}, and neural networks (e.g., \citealt{Simmonds1996, Cabreira2009, Brautaset2020}. Here, three supervised classifiers that take different approaches to classification were compared (Table~\ref{tab:classifiers}). The algorithm k-Nearest Neighbours (kNN; \citealt{Goldberger2004}) was chosen as it has been used for model-informed classification previously \citep{Cotter2021}. LightGBM \citep{Ke2017}, implementation of gradient boosting \citep{Friedman2001}, was considered representative of decision tree-based ensemble methods, with the potential for improved performance compared to random forest \citep{FernandezDelgado2014}, which is widely used in fisheries acoustics \citep{Fernandes2009, Gugele2021}. Finally, the Support vector machine (SVM; \citealt{Cortes1995}) was chosen because it is another widely used algorithm that, together with gradient boosting, has been identified as among the best-performing classification algorithms based on comparisons of performance on large data set collections \citep{FernandezDelgado2014}.

\begin{munlongtab}{|p{1.75cm}|p{5cm}|p{5cm}|p{3.5cm}|}{classifiers}{Overview of the machine learning algorithms compared in this study. The strengths and limitations are detailed for use in fisheries acoustics.}\\
\hline
Classifier & Description & Strengths & Limitations\\
\hline
k-Nearest Neighbours (kNN) & Predicts the class of new samples by taking a majority vote of the k training samples which are closest in distance by some metric (e.g., Euclidean distance) \citep{Fix1951}.  & Few hyperparameters, therefore, easy to implement. Interpretable; an 'explainable artificial intelligence' algorithm \citep{Islam2021}. Computationally inexpensive, facilitates repeat classification and examining the effects of parameters. Limited ability to deal with noise and outliers \citep{Korneliussen2018}. & 
Limited ability to identify low abundance groups \citep{Pena2018}. Vulnerable to overfitting.\\
\hline
LightGBM & Implementation of gradient boosting, a decision tree-based ensemble method similar to random forest \citep{Friedman2001, Breiman2001}. Gradient descent is used to minimise a loss function with the addition of each new tree to the ensemble. Thus, each new tree attempts to correctly classify samples that were previously misclassified \citep{Hastie2009}. & One of the best-performing classification algorithms (e.g., \citealt{FernandezDelgado2014, Zhang2017}).
Suitable for large datasets \citep{Ke2017}, robust to outliers \citep{Hastie2009}.
Reduced risk of overfitting \citep{Hastie2009}. & Rarely used in fisheries acoustics. Many hyperparameters, optimisation is computationally expensive.\\
\hline
Support Vector Machine (SVM) & Maps data to a higher-dimensional feature space in which classes are linearly separable; the optimal decision boundary (hyperplane) has the maximal distance between itself and the closest training data points (support vectors) of any class. \citep{Hastie2009, Cortes1995}. & Few hyperparameters, therefore, easy to implement. Results are consistent and reproducible between repeat implementations \citep{Bennett2000}. & Sensitive to outliers \citep{Kanamori2017}. Unsuitable for large datasets, as it is very computationally expensive \citep{Cervantes2008}.\\
\hline
\end{munlongtab}

\subsubsection{Training on modelled target spectra}
The ML classifier training was completed with Python 3.9 using the Scikit-Learn library (version 1.1.1, \citealt{Pedregosa2011}). An L$^2$-normalization was applied to each target spectra from individuals modelled with the PC-DWBA model simulations so that if the values were to be squared and summed, the sum would equal 1 \citep{Komer2014}. The target variable (i.e., the classification output) was the zooplankton group: copepod, euphausiid, chaetognath, or hydrozoan. \\
For each classifier described in Table~\ref{tab:classifiers}, we optimised its hyperparameters (settings) and estimated its performance on a holdout dataset through cross-validation (CV; \citealt{Stone1974}). Nested CV \citep{Wainer2021} was used to optimise the hyperparameters and evaluate the performance of the classifiers (Appendix~\ref{apdx:SuppMat2} Figure S4). Nested CV ensured that separate data were used to train, validate, and test the classifier and provided an estimate of the classifier's true error with minimal bias \citep{Varma2006}. We compared the classifiers' success using mean class-weighted F1 score (Equation~\ref{eqn:F1}; \citealt{Pedregosa2011}) because that is appropriate for scenarios where both false positives and false negatives are equally undesirable. \\
The F1 score is a measure of overall accuracy calculated as the harmonic mean of precision and recall, defined as:
\begin{muneqn}{F1}
F1 = \frac{2*Precision_i*Recall_i}{Precision_i+Recall_i} = \frac{2*TP_i}{2*TP_i+FP_i+FN_i}.
\end{muneqn}
\\
Precision reports the relative success of the classifier, expressed as:
\begin{muneqn}{precision}
Precision = \frac{TP_i}{TP_i+FP_i},
\end{muneqn}
where TP is the number of true positives and FP is the number of false positives for each class $i$ (each zooplankton group). \\
Whereas recall is a measure of the sensitivity from repeat detections, expressed as:
\begin{muneqn}{recall}
Recall = \frac{TP_i}{TP_i+FN_i},
\end{muneqn}

where FN is the number of false negatives for each class $i$ (each zooplankton group). An F1 score of 1.0 would indicate that a classifier could correctly classify each sample. \\
Hyperparameter optimisation was repeated on the entire modelled dataset (1000 target spectra for each of the four zooplankton groups) without subsampling to obtain the final trained classifiers. 

\subsubsection{Classifier sensitivity}
To determine the optimal frequency bandwidth for model-informed classification of copepods, euphausiids, chaetognaths, and hydrozoans, kNN classifiers were trained and evaluated with modelled target spectra over the bandwidths commonly used in fisheries acoustics \citep{Simmonds2008}. The selected bandwidths were the individual bandwidths from the 70, 120, 200 and 333 kHz transducers produced by Kongsberg Maritime AS (45-90 kHz, 90-170 kHz, 185-255 kHz and 283-383 kHz) and their continuous bandwidth (45-383 kHz). Only kNN was used for this analysis as it is less computationally expensive than the other algorithms.\\
A kNN classifier was also trained using modelled cross-sectional backscattering strength – frequency spectra, $\sigma_{bs}(f)$, the linear scale of TS($f$), for the bandwidth of 185-255 kHz to examine the effect of the logarithmic scale of the modelled target spectra on the classification performance.\\
Additionally, we evaluated the classifiers' sensitivities to the parameterisation of material properties in the scattering models because this can strongly influence backscattering intensity\citep{Chu1999, Sakinan2019}. A PC-DWBA model was parameterised using literature material properties values for Antarctic copepods drawn from the literature (\textit{Calanus} spp.) (\textit{g} = 0.995 $\pm$ 0.001 and \textit{h} = 0.959 $\pm$ 0.010; \citealt{Chu2005}). These values are from spring (2 May 2002) but from similar water temperatures (-0.8 to 0.4 $^{\circ}$C) as those used for Arctic copepods in this study. All other model parameters for copepods and the other zooplankton groups remained the same. \\

\subsection{Classification}
The trained and optimised classifiers were used to classify the measured in situ target spectra from AZKABAN into zooplankton groups (copepods, euphausiids, chaetognaths, or hydrozoans). The classifier predictions were evaluated by comparing: 1) the predicted class distributions to the species composition of the zooplankton sample recovered from AZKABAN; 2) the class predictions from each classifier (classifier agreement); and 3) the class predictions for targets from the same track (within-track consistency). 

\section{Results}
\subsection{Species composition}
The zooplankton sample collected from AZKABAN after the experiment showed that copepods were numerically dominant. Over 20 000 copepods were in AZKABAN, mostly \textit{Calanus} spp. (> 13 000 individuals; Table~\ref{tab:taxonomic}). The second most abundant group was euphausiids, which were an order of magnitude less abundant in the samples than copepods. The most common euphausiid was \textit{Thysanoessa inermis}, and the population consisted mainly of small juveniles (median length of 11mm; Table~\ref{tab:taxonomic}). The sample contained almost as many chaetognaths as euphausiids, predominantly \textit{Parasagitta elegans}. The fourth most abundant group in the sample were hydrozoans, predominantly \textit{Aglantha digitale}. All other zooplankton and fish sampled had < 100 individuals; therefore, we did not include these species in the classification analysis due to the low likelihood of repeated detections. During the experiment, the AZKABAN mesocosm had a total density of 2203 individual zooplankton per m$^3$.

\begin{muntab}{|l|l|r|r|r|}{taxonomic}{Taxonomic group, species, count and proportion of the sample retrieved from the net after the experiment. Samples with <1000 individuals were counted for the entire recovered mesocosm sample. The groups in grey were modelled to create the labelled training dataset for the classification algorithms.}
\hline
Taxonomic group	& Species & Total  & Proportion of  & Median length\\
	&  & individuals &  sample (\%) &  (mm) ($\pm$ SD) \\
\hline
Copepoda & \textit{Calanus} spp. & 13380 & 50.61 & 3.3 ($\pm$ 0.7)\\
\hline
Copepoda & \textit{Metridia} spp. & 6310 & 23.87 & \\
\hline
Copepoda & \textit{Paraeuchaeta} spp. & 710 & 2.69 & \\
\hline
Copepoda & Other copepods & 160 & 0.61 & \\	
\hline
Euphausiacea & \textit{Thysanoessa inermis} & 2485 & 9.40 & 11.0 ($\pm$ 4.0)\\
\hline
Chaetognatha & \textit{Parasagitta elegans} & 2220 & 8.40 & 17.0 ($\pm$ 5.0)\\
\hline
Hydrozoa & \textit{Aglantha digitale} & 1000 & 3.78 & 11.0 ($\pm$ 5.0)\\
\hline
Decapoda & juvenile \textit{Pandalus} spp. & 76 & 0.29 & \\	
\hline
Decapoda & benthic shrimp & 2 & 0.01 & \\
\hline
Pteropoda & \textit{Clione limacina} & 40 & 0.15 & \\	
\hline
Amphipoda & \textit{Themisto} spp. & 27 & 0.10 & \\	
\hline
Amphipoda & Undetermined & 14 & 0.05 & \\	
\hline
Fish (larvae) & \textit{Leptoclinus maculatus} & 7 & 0.03 & \\	
\hline
Mysidacea & Undetermined & 4 & 0.02 & \\
\hline
\end{muntab}

\subsection{Scattering models}
Copepods were the smallest scatterers in this experiment with a median total length ($\pm$ standard deviation) of 3.3 $\pm$ 0.7 mm and an average modelled TS of -113 dB re 1 m$^2$ across the frequency spectrum. The amplitude of the modelled target spectra was typically lower for the copepods than the other three groups. Modelled Antarctic copepods had similar target spectra results but with a 5 dB mean increase across the spectra compared to the Arctic copepods, with an average TS of -107 dB re 1 m$^2$ (Figure~\ref{fig:FigureAZKABAN3}a; blue).\\
Euphausiids and hydrozoans had the same median total lengths of 11 mm ($\pm$  4 mm for euphausiids and $\pm$ 5 mm for hydrozoans). Despite their similar length distributions, euphausiids had a higher average TS (-89 dB re 1 m$^2$ for euphausiids and -94 dB re 1 m$^2$ for hydrozoans) due to differences in their material properties. However, both groups had relatively flat average spectra over the measured bandwidth (Figure~\ref{fig:FigureAZKABAN3}b, d). Lastly, chaetognaths had the longest median length (17 $\pm$ 5 mm) but had a relatively low median TS (-98 dB re 1 m$^2$). The target spectra of chaetognaths had a slight positive slope and a large dispersion of TS intensity (Figure ~\ref{fig:FigureAZKABAN3}c, g).

\munepsfig[scale=.90]{FigureAZKABAN3}{a-d) All PC-DWBA model simulation results for each dominant zooplankton group. For copepods (a) the model results are shown for Arctic species (black; \citealt{Kogeler1987}) and the Antarctic species (blue; \citealt{Chu2005}. e-h) L$^2$-normalised PC-DWBA model simulation results for each dominant zooplankton group.}

\subsection{Noise level}
The noise level inside AZKABAN was low, being below -100 dB re 1 m$^2$ throughout the mesocosm (Figure~\ref{fig:FigureAZKABAN4}) and across the frequency bandwidth. There were peaks in the noise level profile at 1.1 m, 1.6 m and 1.9 m range from the transducer (Figure~\ref{fig:FigureAZKABAN4}). The noise profile followed a similar magnitude and trend across the bandwidth, with approximately 5 dB re 1 m$^2$ variability. We found that the signal-to-noise ratio at 200 kHz was less than 10 dB re 1 m$^2$ for 10.6\% of the single targets used for classification, as shown by the overlaid detected target used for classification analysis in Figure~\ref{fig:FigureAZKABAN4}. This was deemed adequate, and all targets were retained for subsequent analyses.\\

\munepsfig[scale=.90]{FigureAZKABAN4}{Background noise profile inside AZKABAN across the available bandwidth (185-255 kHz; blue lines). The grey dots indicate the TS of each detected tracked target detection at 200 kHz. The detection zone is delimited by the horizontal black lines at 1 m and 2.25 m. The transducer face and top of the net are at 0 m range and the bottom of the net is at 3 m range.}

\subsection{Mesocosm target detections}
A total of 7,722 tracked single targets were detected during the three-hour AZKABAN mesocosm experiment. The mesocosm target detections were from a mixed zooplankton assemblage, and individual detections were from targets of unknown identity. There were 777 distinct tracks, with a mean of 10 single target detections per track. The minimum number of detections in a track was 4, and the maximum was 178. 

\subsection{Evaluation of classifier training}
The optimised kNN classifier used the KDTree algorithm \citep{Pedregosa2011} and Euclidean distance as the distance metric. For the kNN classifier, the optimised value for the number of training samples closest in distance to the query sample used for predictions, k, was 1. The optimised SVM classifier used a radial basis function kernel, and the optimised LightGBM comprised 3,400 trees with a maximum tree depth of seven. Full details of the optimised classifiers are provided in the Appendix~\ref{apdx:SuppMat2} Code S1, S2 and S3.

\subsubsection{Classifier performance}
The F1 scores reflect the classifiers' performance at classifying the modelled target spectra. The highest class-weighted F1 score was achieved using LightGBM (0.71 $\pm$ 0.02), followed by kNN (0.70 $\pm$ 0.03) and SVM (0.59 $\pm$ 0.03) for the 185-255 kHz bandwidth. Per-class F1 scores showed consistently highest scores for copepods (0.71-0.87). The lower per-class F1 scores for euphausiids (0.64-0.72), hydrozoans (0.58-0.67) and chaetognaths (0.44-0.58) indicated that the classifiers had limited precision and/or recall in classifying these groups. The limited precision and recall of the classifiers were reflected in the confusion matrices for each classifier (i.e., the high numbers of misclassifications; Appendix~\ref{apdx:SuppMat2} Table S3, S4 and S5). \\

\begin{muntab}{|l|l|l|l|}{F1}{Classifier mean F1 scores estimated through nested cross-validation (mean $\pm$ standard deviation) for the 185-255 kHz bandwidth. A score of 1.0 indicates that a classifier could correctly classify each sample (100\% classification success).}
\hline
Classifier & kNN & LightGBM & SVM\\
\hline
\hline
Class-weighted & 0.70 $\pm$ 0.03 & 0.71 $\pm$ 0.02 & 0.59 $\pm$ 0.03\\
\hline
Copepods & 0.87 $\pm$ 0.02 & 0.87 $\pm$ 0.02 & 0.71 $\pm$ 0.03\\
\hline
Euphausiids & 0.70 $\pm$ 0.03 & 0.72 $\pm$ 0.03 & 0.64 $\pm$ 0.03\\
\hline
Chaetognaths & 0.58 $\pm$ 0.04 & 0.58 $\pm$ 0.05 & 0.44 $\pm$ 0.03\\
\hline
Hhydrozoans & 0.66 $\pm$ 0.04 & 0.67 $\pm$ 0.03 & 0.58 $\pm$ 0.04\\
\hline
\end{muntab}

\subsubsection{Classifier sensitivity}
The nested CV procedure was conducted for modelled target spectra across five different frequency bandwidths (45-90 kHz, 90-170 kHz, 185-255 kHz, 283-383 kHz, and 45-383 kHz) to test the effect of bandwidth selection on classifier performance. The comparisons were only run with kNN because it was the least computationally expensive algorithm of those used in this study and, based on the results in Table~\ref{tab:F1}, provided similar performance to LightGBM. The mean class-weighted F1 score for kNN with the full bandwidth (TS$_{45-383 kHz}$) was 0.92 ($\pm$ 0.02) (Appendix~\ref{apdx:SuppMat2} Table S6). The best score for a single ``transducer" was 0.86 ($\pm$ 0.01), using modelled spectra at the centre bandwidth of the 120 kHz transducer (TS$_{70-190 kHz}$). \\
The cross-sectional backscatter spectra ($\sigma_{bs_{185-255 kHz}}$) (i.e., the linear domain representation of the target spectra) were also used to train a kNN classifier. Using the linear scale of the target spectra brought a slight improvement to classifier performance (mean class-weighted F1 score: 0.73 $\pm$ 0.03 in the linear domain compared to 0.70 $\pm$ 0.02 in the logarithmic domain).\\
The performance of the kNN classifier trained with modelled target spectra of Antarctic copepods (Appendix~\ref{apdx:SuppMat2} Code S5) (mean class-weighted F1 score: 0.69 $\pm$ 0.03; Appendix~\ref{apdx:SuppMat2} Table S7) was not significantly different from the classifier trained with modelled target spectra of Arctic copepods (mean class-weighted F1 score: 0.70 $\pm$ 0.02).

\subsection{Classification of \textit{in situ} measurements}
All classifiers predicted a different class distribution to the species composition of the zooplankton sample recovered from AZKABAN (Figure~\ref{fig:FigureAZKABAN5}). For kNN, hydrozoans were predicted to be the most abundant class, followed by chaetognaths, euphausiids, and copepods, which was the inverse of the recovered sample (Figure~\ref{fig:FigureAZKABAN5}). For LightGBM, chaetognaths were predicted as the most abundant class with no copepod detections. The SVM predictions implied a majority of hydrozoans, followed by euphausiids, chaetognaths, and copepods.\\

\munepsfig[scale=.75]{FigureAZKABAN5}{a) Composition of the zooplankton sample used in the mesocosm experiment as a proportion of the total sample for the four most abundant groups (n=26,435). b-d) the proportion of predicted targets of the total detections for tracked single targets (n=7,722) assigned to each group by k-nearest neighbours (kNN), LightGBM and support vector machine (SVM) classifiers.}

The measured \textit{in situ} target spectra for each class, as classified by kNN and LightGBM, were generally consistent with each other and the modelled spectra (Figure~\ref{fig:FigureAZKABAN6}). However, the measured \textit{in situ} target spectra classified as copepods by kNN had a higher target strength than the copepods' modelled target spectra (Figure~\ref{fig:FigureAZKABAN6}). Of the mesocosm targets, those with high intensity and flat target spectra were labelled as copepods by the SVM classifier. However, the target spectra for euphausiids, chaetognaths, and hydrozoans predictions from SVM were in general agreement with the modelled results.\\

\munepsfig[scale=.75]{FigureAZKABAN6}{\textit{Modelled}: PC-DWBA model simulations (theoretical) target spectra for each zooplankton group. \textit{kNN, LightGBM}, and \textit{
SVM}: measured target spectra of tracked single targets from the mesocosm experiment as classified by k-Nearest Neighbours (kNN), LightGBM, and support vector machine (SVM). All panels include the number (n) of target spectra in each panel..}

Only 18.13\% of the measured target spectra (1,400 samples) were classified as the same zooplankton group by all three classifiers: 10.09\% were consistently classified as hydrozoans, 5.93\% as chaetognaths, 1.29\% as euphausiids, and 0\% for copepods because no target spectra were labelled as copepods by LightGBM. Pairwise comparisons of classifiers show that 50.62\% of tracked single target spectra (3,909 samples) were classified as the same zooplankton group by kNN and LightGBM, compared to 42.55\% (3,286 samples) by kNN and LightGBM, and 29.31\% (17,103 samples) by LightGBM and SVM. \\
SVM had the highest within-track prediction consistency: on average, 75\% of targets within a track were assigned the same class label. However, 70\% of tracks included at least two different classes. For LightGBM, 67\% of detections within a track were assigned to the same class, and 100\% of tracks included at least two classes, compared to 62\% and 93\%, respectively, for kNN.

\section{Discussion}
\subsection{AZKABAN: A mesocosm for \textit{in situ} broadband acoustic backscatter measurements}
AZKABAN was designed to facilitate \textit{in situ} broadband acoustic backscatter measurements of caged fish and zooplankton. The estimated noise level of AZKABAN was sufficiently low to enable the detection of mesozooplankton. Noise and reverberation from mesocosm walls have been a major challenge in past experiments with weak scatterers \citep{Knutsen1997}. The successful detection of weak targets in the AZKABAN mesocosm was partly due to the improvements in signal-to-noise ratio and range resolution associated with pulse compression of the broadband received signal.\\
The purpose-built mesocosm offered a practical platform for broadband measurements of mesozooplankton. The design enabled the zooplankton sample to be added to the submerged net from a small boat, minimising stress on the animals. It was also possible to recover the samples after the experiment for enumeration and morphometric analysis. Therefore, this mesocosm could be an effective experimental setup for controlled behavioural experiments, such as reactions to different sources and intensities of light and sound.

\subsection{Performance of classifiers trained by modelled target spectra}
Of the three conceptually different classifiers trained on modelled target spectra, the best-performing classifier was LightGBM, with a mean class-weighted success rate of 0.71. Copepods consistently had the highest mean F1 score (0.71-0.87), indicating that copepods' modelled target spectra could be discriminated from the others. The sensitivity analysis with the copepods parametrised with Arctic or Antarctic material properties demonstrated that changes in \textit{g} and \textit{h} have little effect on the normalized target spectra (Figure~\ref{fig:FigureAZKABAN3}a,e) or classification success (Table~\ref{tab:F1}, Appendix~\ref{apdx:SuppMat2} Table S7). All the classifiers were limited in their ability to discriminate between euphausiids, chaetognaths and hydrozoans. Despite parametrising the scattering models with representative parameters and shapes of the different zooplankton group organisms, these groups had overlapping modelled target spectra. Presumably, the overlap in the modelled target spectra of euphausiids, chaetognaths and hydrozoans is due to the close similarity of the parameter distributions. The model's inability to resolve the target spectra of different fluid-like zooplankton directly introduces consequences for target detection and classification. This suggests that thresholds should be established to determine possible taxonomic resolution for classification of species with overlapping model parameter distributions. \cite{Ross2013} report a similar effect with juvenile euphausiids and pteropods and conclude that the similarity in frequency responses of these groups may render them indistinguishable. \\
Previous studies on supervised classification of target spectra have used coarse taxonomic resolution to manually label measured target spectra to create a training set based on model-informed classes. \citep{Cotter2021} achieved a class-weighted F1 score of 0.90 for the classification of manually labelled fluid-like and gas-bearing targets detected with a broadband echosounder (25-40 kHz) using k-Nearest Neighbours. \citep{Roa2022} classified six reef fish using scattering models with a wide bandwidth (30-200 kHz) and found high classification accuracy (F1-score > 80\%). We also found a wide bandwidth (45-383 kHz) resulted in high classification performance (class weighted mean F1-score of 0.92 $\pm$ 0.02). However, the wide bandwidth (45-383 kHz) results were not possible to validate with the mesocosm experimental setup. Furthermore, we achieved higher accuracy with a lower bandwidth (90-170 kHz; class weighted mean F1-score of 0.86 $\pm$ 0.01) than the one used in the mesocosm (185-255 kHz). Despite the higher F1-scores at a lower bandwidth (90-170 kHz), we used 185-255 kHz for its smaller wavelength for a better target size resolution. For the classification of \textit{in situ} measurements, physical and practical limitations of target size and echosounder properties (beamwidth, wavelength, transmit power) must be considered in addition to the F1-score of the classifier. While previous model-informed classification studies \citep{Cotter2021, Roa2022} may have achieved better classification performances because the classes they used had distinct acoustic properties, in contrast to our study their model-informed classifiers were not validated with \textit{in situ} measurements.

\subsection{Discrepancies between classifiers predictions and i\textit{n situ} measurements}
We used the AZKABAN mesocosm experiment to validate the performance of three model-trained classifiers using measurements of a mesozooplankton community sample for which the species composition was known. Overall, the zooplankton community composition determined by the classifiers differed from the actual composition in the mesocosm. Copepods were overwhelmingly the most abundant group in the mesocosm (Figure~\ref{fig:FigureAZKABAN5}a) but were consistently the least abundant class in the classifier predictions (Figure~\ref{fig:FigureAZKABAN5}b-d). It is possible that the small copepods in the mesocosm (3 mm length) were not detected, given the spatial resolution of the wavelength (7 mm at 200 kHz; \citealt{Simmonds2008}. Classifier predictions reflected this, with few copepod predictions and the target strength mismatch between the modelled and predicted results (Figure~\ref{fig:FigureAZKABAN6}) despite the relatively high F1-scores for copepods. Hydrozoans were the least abundant group in the mesocosm but the most abundant predicted class for kNN and SVM. Whereas for LightGBM, chaetognaths were the most abundant class. \\
The \textit{in situ} target detections were also used to assess the within-track consistency of predictions between classifiers. The target tracking algorithm associates many single target detections to an individual organism as it travels across the acoustic beam. There was a high variability of zooplankton groups assigned to each track, highlighting the large variability in target spectra from an individual organism. The inhomogeneity of predictions per track and poor agreement between classifiers provide compelling evidence that model-informed classification of fluid-like mesozooplankton is unreliable. 

\subsection{Recommendations to improve model-informed classification of zooplankton}
Our results suggest that the model-informed classification is highly dependent on the choice of the classifier when the groups cannot be reliably differentiated. Good practice for machine-learning-based science typically requires that a classifier's performance is evaluated on a test set `drawn from the distribution of scientific interest' \citep{Kapoor2022}. Model-informed acoustic target classification is appealing because it avoids the practical challenges and cost of obtaining labelled measurements of known species empirically (by sampling in the field or tank). However, using model-informed classification inevitably means that the samples used to train, validate, and test a classifier are not drawn from the distribution of scientific interest. \\
This study used a scattering model flexible to geometry, material properties, and acoustic frequency changes to generate training data for supervised machine learning classifiers. For future studies, we suggest that model-informed classification could be useful in assessing the theoretical classification potential of different bandwidths. However, classifier performance must be considered in the context of factors such as the target strength of the species of interest at a given frequency and the frequency's range resolution for the classification of \textit{in situ} measurements. We conclude that a better understanding of the variability in the acoustic measurements from individuals is required before model-informed classification of target spectra can be implemented reliably. Features in broadband spectra, such as the locations of nulls and peaks, can provide insight into morphological characteristics of individuals \citep{Reeder2004, Kubilius2020}. A better understanding of these features could increase classification potential and the information we can extract from target spectra.

\section{Summary and conclusions}
This study evaluated a model-informed classification of zooplankton from broadband echosounder data using \textit{in situ} measurements (185 to 255 kHz) of a mixed Arctic mesozooplankton assemblage in a purpose-built mesocosm. Acoustic scattering models generated modelled target spectra for the four most abundant zooplankton groups in the mesocosm: copepods, euphausiids, chaetognaths and hydrozoans. Three different supervised machine learning algorithms were trained using modelled target spectra, and then compared in terms of their ability to classify the \textit{in situ} measured target spectra obtained from the mesocosm experiment. Investigations showed that kNN and LightGBM classifiers could differentiate copepods from the other groups reasonably well, but that they could not differentiate euphausiids, chaetognaths, and hydrozoans reliably. The classifier training results were confirmed by their inconsistent predictions within-track and between classifiers for the \textit{in situ} mesocosm measurements. The lack of consistent predictions within a track suggests that the variability in target spectra per class is greater than in the target spectra between the different zooplankton groups from the sound scattering models. The outstanding challenge remaining is to understand the ping-to-ping variability in the spectra of individual scatterers \citep{Martin1996, Dunning2023}.\\
The mesocosm design used in this study was an effective platform for measurements of fluid-like scatterers, which could be used to develop a better understanding of measured variability in target spectra. For example, mesocosm experiments with fewer individuals or a series of single species experiments could improve model validation for broadband echosounder measurements of freely swimming individuals. However, a semi-permanent installation for longer experiment periods, visual validation through video or imaging for swimming behaviour information and repeat experiments would be required to complete such comparative studies. 


