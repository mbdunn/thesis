\chapter{Classification of three coinciding species (Atlantic cod, polar cod, and northern shrimp) using broadband hydroacoustics}
\label{chap:classification}



Muriel Dunn$^{1,2}$, Geir Pedersen$^3$, Malin Daase$^{4,5}$, Jørgen Berge$^{4,5}$, Emily Venables$^{4,}$, S\"{u}nnje L. Basedow$^4$, Stig Falk-Petersen$^6$, Jenny Jensen $^1$, Tom J. Langbehn$^7$, Lionel Camus$^1$, and Maxime Geoffroy$^{2,4}$\\

$^1$ Akvaplan-niva AS, Fram Centre, Postbox 6606, Stakkevollan, 9296 Tromsø, Norway \\
$^2$ Center for Fisheries Ecosystems Research, Fisheries and Marine Institute of Memorial University of Newfoundland and Labrador, St. John's, A1C 5R3, NL, Canada\\
$^3$ Institute for Marine Research, 5005 Bergen, Norway\\
$^4$ Department of Arctic and Marine Biology, UiT The Arctic University of Norway, 9019 Tromsø, Norway\\
$^5$ Department of Arctic Biology, The University Centre in Svalbard, 9171 Longyearbyen, Norway \\
$^6$ Independent scientist, 9012 Tromsø\\
$^7$ Department of Biological Sciences, University of Bergen, 5020 Bergen, Norway\\



Submitted to \textit{ICES Journal of Marine Sciences} (29 August 2023) \\

\section{Abstract}
The northern shrimp (\textit{Pandalus borealis}) fishery, an important commercial fishery in the eastern Canadian Arctic and the Barents Sea, is reporting increasing bycatch of polar cod (\textit{Boreogadus saida}), a key Arctic forage fish species. Furthermore, northern shrimp and polar cod spatial distributions increasingly coincide with that of Atlantic cod (\textit{Gadus morhua}). Discrimination between the acoustic signals of Atlantic cod, polar cod and northern shrimp could provide more information on the risk of polar cod bycatch in the northern shrimp fishery and improve the accuracy of stock assessment surveys. We conducted a series of single-species mesocosm experiments for target strength measurements of Atlantic cod, polar cod and northern shrimp to assess the potential for species discrimination using their target strength spectra, TS(f). Mesocosm experiments were completed with a Wideband Autonomous Transceiver (WBAT) and collected broadband TS(f) (90-170 kHz and 185-255 kHz) of individual targets. Hundreds of TS(f) were extracted for each species and used to train machine-learning classification algorithms (``classifiers"). We found that a simple classifier, k-Nearest Neighbor, and a single 200 kHz transducer operating in broadband mode are sufficient to achieve high classification performance (97\%). The promising results from mesocosm-trained acoustic classifiers are an important step towards classifying coinciding marine species \textit{in situ} and increasing the sustainability of fisheries. 

\section{Introduction}
The northern shrimp fishery, \textit{Pandalus borealis}, is one of the most valuables fisheries in the Northwest Atlantic, the eastern Canadian Arctic, and the Barents Sea. It generates 90\% of Greenland's export value \citep{Garcia2007}, and is the most valuable invertebrate fishery in the Barents Sea \citep{Berenboim2000}. However, shrimp fisheries are associated with bycatch problems \citep{Howell1992, Grimaldo2005}. Currently, bycatch reduction strategies in the shrimp trawl fishery focus on gear modification, such as rigid sorting grids \citep{Hannah2007}, which has a limited effect on separating small gadoids \citep{Isaksen1992, Grimaldo2005}, in particular Atlantic cod (\textit{Gadus morhua}) and polar cod (\textit{Boreogadus saida}) \citep{Walkusz2020}. Atlantic cod is an important predator of northern shrimp and influences the stock size \citep{Garcia2007}. Polar cod can account for > 95\% of the pelagic fish assemblage in the Arctic and has a pivotal role in the Arctic food web as a key forage fish species \citep{Geoffroy2023}. A better understanding of the spatial dynamics of Atlantic cod, polar cod, and northern shrimp, would help forecast stock changes and bycatch risk. 
Hydroacoustic surveys are widely used for expanding the spatial footprint of ecosystem assessments \citep{Bassett2018}. They support high-resolution fish abundance, biomass, and movement estimates and are less invasive than traditional trawl monitoring \citep{Trenkel2019}. Because the acoustic scattering of a target, which depends on size, orientation, and acoustic properties, is also dependent on frequency, broadband echosounders have been increasingly used to infer species composition in hydroacoustic surveys (e.g., \citealt{Ross2013, Loranger2022}). However, broadband acoustic scattering measurements of an individual target, conventionally recorded as target strength spectra (TS(f); hereby target spectra), has been found to have large variability, which cannot be explained by length or orientation \citep{BrisenoAvena2015, Bassett2018, Dunning2023}. 
The increased variability, complexity and size of broadband data have required powerful data analysis tools, such as machine learning algorithms \citep{Malde2020}. However, supervised machine learning algorithms require training datasets containing measurements of known targets. Schooling pelagic fish have been a common test for acoustic classification of monospecific aggregations (e.g., \citealt{Bassett2018, Brautaset2020, Fernandes2009} because of the possibility of validation by trawl. Less invasive measures applicable to individual targets, such as camera validation \citep{BenoitBird2020}, have successfully classified acoustic signals. However, these methods are limited by the ability to identify the acoustically detected individuals when multiple species are present. Mesocosm classification of target spectra has been successfully used to differentiate between two conincident swim-bladdered fish species: whitefish (\textit{Coregonus wartmanni}) and stickleback (\textit{Gasterosteus aculeatus}) \citep{Gugele2021}. A mesocosm-trained classification approach represents a promising avenue to improve taxonomic resolution from broadband hydroacoustics because a high number of detections can be collected for a known population with semi-natural swimming behaviours for a range of different species.
This study reports on a series of single-species mesocosm experiments with broadband hydroacoustics to classify the acoustic backscatter from three coinciding species: Atlantic cod, polar cod, and northern shrimp. In addition, these experiments further improve our understanding of the potential use of mesocosm-trained classification of broadband acoustic backscatter. For the shrimp fishery, classification of cooccurring species with high risks of bycatch using broadband hydroacoustic surveys could help in assessing fishing grounds for a forecast of bycatch risk prior to setting the trawls or to inform policy and models on ecosystem distribution patterns. Remote target classification with broadband acoustics could also benefit stock assessment surveys and estimates by increasing spatial resolution, access to remote areas, and sustainability by reducing survey time and costs.


\section{Methods}
\subsection{Species collection}
The three coinciding species (Atlantic cod, polar cod, and northern shrimp) were collected from R/V \textit{Helmer Hanssen} using a Harstad pelagic trawl (8 mm mesh) and bottom trawl (Campelan 1800 shrimp trawl with rockhopper gear) at 3 knots for 15 to 20 minutes in three fjords in Svablbard (Billefjorden, Krossfjorden, and Konsgsfjorden) (Figure~\ref{fig:FigureAFKABAN1}) on 17 and 19 January 2023 (Table~{\ref{tab:collection}). The trawled depth was selected based on the depth of the strongest scattering layer seen on the vessel's echosounder (Kongsberg Maritime AS; Simrad EK60, 18 and 38 kHz, 1.024 ms pulse duration, 2 Hz pulse repetition). A FISH-LIFT, an aquarium attached to the trawl codend that reduces turbulence and minimises the impact of trawling on the caught animals \citep{Holst2000}, was used to maximise the fitness and health of the live fish and shrimp. The fish and shrimp were kept on board in large tanks (1 m$^3$) with running seawater and delivered to the wharf in Ny-Ålesund in Kongsfjorden. At the Kings Bay Marine Laboratory, the fish and shrimp were stored in 6 m$^3$ holding tanks with a flow-through system of filtered ambient seawater ($\sim$1$^{\circ}$C) for 2 to 7 days, depending on weather and experiment priority.\\

\begin{muntab}{|r|l|r|r|l|r|}{collection}{Overview of the trawling when the catch was used for mesocosm experiments. The pelagic trawl was used unless otherwise noted. All dates are in 2023.}
\hline
Trawl date & Location & Collection & Experiment & Species (n) & Experiment \\
(UTC) & ($^{\circ}$N, $^{\circ}$E) & depth (m) & date &  & duration (h) \\
\hline
17 Jan & Billefjorden & 102 & 19 Jan & Polar cod (90) & 6 \\ \cline{4-6}
22:26 & (78.62, 16.54) &  & 24 Jan & Polar cod (133) & 6.5 \\
\hline
19 Jan & Outer & 150 & 26 Jan & Northern & 5.25 \\
00:10 & Krossfjorden & & &  shrimp (100) & \\ \cline{4-6}
& (79.05, 11.35) &  & 20 Jan & Atlantic cod (5) & 8 \\ \cline{1-3} 
19 Jan & Outer & 352* & & Atlantic cod (11) & \\ 
19:54 &  Kongsfjorden & & & & \\
& (79.04, 11.34) & & & &\\
\hline
\end{muntab}
\begin{flushleft}
* Bottom depth – bottom trawl was used.\\
\end{flushleft}

\munepsfig[scale=.75]{FigureAFKABAN1}{A) Map of the Svalbard archipelago. B) Map of two species collection location (blue circles) and experiment location at the Ny-Ålesund wharf (red square) in Outer Krossfjorden and Outer Kongsfjorden. C) Map of a species collection location in Billefjorden in relation to Longyearbyen (red square).}


\subsection{Mesocosm experiment}
Broadband target strength data of single species were collected during four experiments in January 2023 (Table~\ref{tab:collection}) using a mesocosm deployed from a wharf in Ny-Ålesund, Svalbard (red square; Figure~\ref{fig:FigureAFKABAN1}B). The mesocosm, or AFKABAN (Arrested Fish Kept Alive for Broadband Acoustics Net experiment), was fitted with a large cuboid fish net (H7 x W2 x L2 m) with a 6 mm by 3 mm oval mesh or a small cuboid zooplankton net (H3 x W2 x L2 m) with a 500 µm-mesh (Figure~\ref{fig:FigureAFKABAN2}A). The net was mounted on an 8 m high by 2 m wide and 2 m long aluminium frame oriented vertically (Figure~\ref{fig:FigureAFKABAN2}A). Ropes with hook and loop straps attached the eyelets on the net to the frame at each corner and along the edges. A zipper on the top panel was opened to introduce species into the submerged mesocosm. \\ 
The transducers (ES120-7CD and ES200-7CDK-split; Kongsberg Maritime AS, Horten, Norway) were mounted side by side on a plate centred inside the mesocosm through a hole on the top panel of the net with the acoustic axis pointing directly down. The smaller transducer (ES200-7CDK-split) was mounted on raisers to level the transducer faces. The transducer plate was fixed to the frame to ensure the transducer, the frame, and the net moved as a unit under the stress of currents. AFKABAN was suspended from a crane and lowered into the sea (Figure~\ref{fig:FigureAFKABAN2}B) until the depth of the transducer face was approximately 1 m below the surface. A Wideband Autonomous Transceiver (WBAT, SN:253120; Kongsberg Maritime AS) was fastened horizontally to the frame to operate the transducers (Figure~\ref{fig:FigureAFKABAN2}). The AFKABAN frame was purpose-built by Havbruksstasjonen (Ringvassøya, Norway) and wide enough to have two side-by-side beams of 7$^{\circ}$ opening angle transducers inside the net. \\
The acoustic data were collected using a WBAT programmed to emit frequency-modulated chirps multiplexing between bandwidths 90-170 kHz and 185-255 kHz. The emitted pulse had a fast taper, a pulse duration of 512 μm with 200 and 113 W emitted power for the 120 kHz and 200 kHz transducers, respectively. The ping interval was set to the minimum allowable value, between 2 and 2.5 s, to maximise the number of single detections and tracks; it was limited by factors such as the internal processing time and range. We selected a fast taper to have the maximal bandwidth available at full power for the classifier. A short pulse length was selected to resolve targets near the net boundary, reduce reverberation volume \citep{Soule1997}, and increase the chances of sampling clean echoes from single targets in the mesocosm \citep{Gugele2021}. Data collection for analysis started at least 25 minutes after the mesocosm was fully submerged with the species inside the net to leave enough time for the organisms to settle and bubbles to disperse. 
Continuous conductivity, temperature, and pressure data measurements were collected during all experiments with a Sea-Bird SBE19plus (SN 01908096) for the fish experiments and SeaBird 37SI MicroCAT CTD (SN 37SI31215-2767) for the northern shrimp experiment.\\
Immediately after the experiment, the frame was lifted to the wharf and the species were removed from the net via a zipper on the bottom panel. The shrimp and fish were euthanised in an overdose of Finquel MS-222 (tricaine methane sulfonate) compound solution (500-600 mg l-1). Length and weight measurements were taken on the euthanised individuals after the experiment. The treatment and use of species in these experiments were approved by the Norwegian Food Safety Authority (FOT 29801, 22/231325) (Appendix ~\ref{apdx:Ethics1}).


\munepsfig[scale=.85]{FigureAFKABAN2}{A) Schematic of the frame with the small zooplankton net (left; northern shrimp experiment) and large fish net (right; Atlantic cod and polar cod experiments). The acoustic transceiver (yellow cylinder) is attached to the frame and the transducers (orange cylinder, two in this experiment). There is a hole at the top of the net for the transducer faces to be unobstructed inside the net. B) The AFKABAN mesocosm with the large fish net lifted with the crane at the end of the experiment.}



\subsection{Acoustic data analysis}
Acoustic data were calibrated using the standard sphere method \citep{Demer2015}. The calibrations required two spheres for each transducer (38.1 mm and 22 mm) to collect calibration parameters for the available frequency bandwidths (Appendix~\ref{apdx:SuppMat3} Figure S1). Calibrations were performed on 26 January 2023 in Ny-Ålesund, Svalbard.
All acoustic data were processed in Echoview 13.1 (Echoview Software Pty Ltd, Hobart, Tasmania). The data analysis range was bounded by the near-field region \citep{Simmonds2008}, and by the echo from the bottom of the net (i.e., 1.0 m - 6.8 m for the fish experiments and 1.0 m - 2.4 m for the shrimp experiments). The ``Single target detection – wideband 1" operator was applied to select qualifying targets for each transducer (Table~\ref{tab:SED}). The target strength, TS, threshold was adjusted for the different experimental species; all other parameters were consistent between experiments.\\

\begin{muntab}{|l|r|}{SED}{Single echo detection - wideband 1 detector settings, where TS is target strength.}
\hline
Parameter & Value \\
\hline
TS threshold (dB re 1 m$^2$) & fish: -75  \\
& shrimp: -120\\
\hline
Pulse length determination level (dB re 1 W$^2$) & 8\\
\hline
Normalised pulse length (min, max) & (0.5, 1.5)\\
\hline
Minimum target separation (m) & 0\\
\hline
Maximum beam compensation (dB re 1 m$^2$) & 191\\
\hline
Off-axis angle filter (degrees) & 4\\
\hline
\end{muntab}
\begin{flushleft}
$^1$ \citep{Dunning2023}.\\
\end{flushleft}

All the single targets accepted by single echo detection (SED) parameters of both transducers were merged for manual target selection. The selected single echo detections were manually organised into tracks by visual assessment to ensure each track had a high probability of being from an single organism \citep{Khodabandeloo2021}. We selected isolated SEDs that did not contain adjacent targets in the Fourier transform window (0.25 m above and below) (Figure~\ref{fig:FigureAFKABAN3}D-F). Adjacent targets can distort the frequency response because of interference between the backscattered signals. Target spectra graphs (Figure~\ref{fig:FigureAFKABAN3}G-I) were used to assess the presence of adjacent targets, these can be identified by regularly spaced nulls \citep{Stanton1996, Reeder2004, Khodabandeloo2021}. The single target tracks were formed by following SEDs traces from ping to ping and verifying with the location sequence of the single target tracks across the acoustic beam. We ensured each selected track had a minimum of 4 SEDs to have enough information for the target trajectory across the acoustic beam. Only one SED per ping could be selected, in the case of multiple SED candidates in a single ping, the center, strongest SED was selected for the track. A single organism likely formed each track because we used both frequency response patterns and target tracking location in the acoustic beam to select targets and create tracks (Figure~\ref{fig:FigureAFKABAN3}). \\

\munepsfig[scale=.75]{FigureAFKABAN3}{Examples of the target spectra of each selected detection from an individual track of each species using multiplexing broadband echosounders. An image of the species from the experiment; A) Atlantic cod (\textit{Gadus morhua}), B) polar cod (\textit{Boreogardus saida}), C) northern shrimp (\textit{Pandalus borealis}). D-F) Echogram of a selected isolated track from each species labeled above. Measured target spectra of the selected tracks; G) Atlantic cod with 19 detections in the 94-158 kHz bandwidth and 13 detections in the 189-249 kHz bandwidth (grey lines), H) polar cod track with 6 in the 94-158 kHz bandwidth and 4 target spectra in the 189-249 kHz bandwidth, I) northern shrimp track with 9 in the 94-158 kHz bandwidth and 10 target spectra in the 189-249 kHz bandwidth.}

All target spectra were calculated using a Fourier transform window length of 0.33 times the pulse length (i.e., 0.25 m) and exported from Echoview with a 2 kHz frequency resolution, determined by the pulse duration \citep{Medwin1998, Khodabandeloo2021}. The Fourier transform window size was selected to maximise the information from the echo while reducing the risk of contamination from nearby targets.
The first and last 5\% of each target spectra were removed to eliminate the effects of the pulse taper. The frequency band from 162-170 kHz was removed because of inconsistent calibration results at this frequency range (Appendix~\ref{apdx:SuppMat3} Figure S1). The trimmed target spectra were used to train the classifiers.\\

\subsection{Classifier training}
Classifier training was performed in Python (version 3.9.15) using the Scikit-Learn library (version 1.1.3, \citealt{Pedregosa2011}) and Hyperopt-Sklearn library (version 1.0.3; \citealt{Komer2014}). A L$^2$-normalization was applied to each target spectra so that if the values were to be squared and summed, the sum would equal 1 \citep{Komer2014}. Preprocessing with normalizing by observation removed the influence of intensity on the off-axis compensation of the target spectra. The number of target spectra per class was balanced by applying an over-sampling technique because the classes were not severely unbalanced (6:1) and to avoid removing samples. The samples in the minority classes (Atlantic cod and northern shrimp) were resampled randomly until they were balanced with the majority class (polar cod) to reduce the risk of bias in the model predictions \citep{Goodfellow2016}, reaching a total of 695 and 699 samples per class for the 94-158 kHz and 189-249 kHz bandwidths, respectively.\\
Three classifiers, K-Nearest Neighbours (kNN; \citealt{Goldberger2004}), LightGBM \citep{Ke2017}, and support vector machine (SVM; \citealt{Cortes1995}), were trained and Bayesian hyperparameter optimisation was used for parameter selection. The classifiers were trained using a 10-fold cross-validation method \citep{Stone1974} to split the data iteratively into a training subset (10\%) and a testing subset (90\%)  of the single species target spectra from the mesocosm experiments. Classifier performance was evaluated using a mean class-weighted F1 score because it is an evaluation metric that penalises false positives and false negatives equally. The class-weighted F1 score was averaged by class and weighted by the number of true instances for each class \citep{Pedregosa2011}.

\section{Results}
\subsection{Species composition}
The 16 Atlantic cod in AFKABAN had a mean length of 52 $\pm$ 8 cm (L $\pm$ standard deviation (SD)), and their mean weight was 978 $\pm$ 346 g (W $\pm$ SD). The individuals were smaller for both polar cod experiments than for the Atlantic cod experiment. The first polar cod experiment had fewer but larger individuals (n = 90; L = 19 $\pm$ 2 cm; W = 50 $\pm$ 10 g), whereas the second experiment had more individuals that were, on average, smaller (n = 133; L = 18 $\pm$ 2 cm; W = 30 g; weighed as a group and divided by the number of individuals). For the shrimp experiment, we added 100 shrimps with an average length of 8 $\pm$ 1 cm (measured from eye to telson) inside the small mesocosm configuration of AFKABAN (Figure~\ref{fig:FigureAFKABAN1}A left). The shrimps were weighed as a group and divided by the number of individuals, which resulted in an average individual weight of 6 g.

\subsection{Single species target spectra}
There were 60 selected tracks in the Atlantic cod dataset, comprised of 345 target spectra in the 94-158 kHz bandwidth and 273 target spectra in the 189-249 kHz bandwidth (Figure~\ref{fig:FigureAFKABAN4}A). The first polar cod experiment resulted in 62 selected tracks with 345 target spectra in the 94-158 kHz bandwidth frequency bandwidth and 362 target spectra in the 189-249 kHz frequency bandwidth (Figure~\ref{fig:FigureAFKABAN4}B). The second polar cod experiment was slightly shorter in length but had more individuals in the net (Table~\ref{tab:collection}) and had slightly more tracks, a total of 66 tracks with 350 target spectra in the 94-158 kHz bandwidth and 337 target spectra in the 189-249 kHz bandwidth (Figure~\ref{fig:FigureAFKABAN4}C). Lastly, the northern shrimp experiment had the fewest tracks because of the low number of individuals, the shorter duration of the experiment and the small size of the individuals (Table~\ref{tab:collection}). There were 25 selected tracks composed of 108 target spectra in the 94-158 kHz bandwidth and 180 target spectra in the 189-249 kHz bandwidth (Figure~\ref{fig:FigureAFKABAN4}D). \\

\munepsfig[scale=.85]{FigureAFKABAN4}{Target spectra of all single target detections from all single species experiments. Each target spectra was recorded as target strength (TS) over the 94-158 kHz or the 189-249 kHz bandwidth. \textit{Panel A-D:} target spectra of single echo detections organised by species. \textit{Panel E-H}: L$^2$-normalized target spectra, each target spectra have a unit norm.}

The classification analysis used all the selected target spectra and, in the case of Atlantic cod and northern shrimps, the replicates added to achieve balanced classes (Figure~\ref{fig:FigureAFKABAN4}). Atlantic cod had the strongest average echo intensity with a mean target strength of -34 dB re 1 m$^2$ for the 94-158 kHz bandwidth and -38 dB re 1 m$^2$ for the 189-249 kHz bandwidth. Both polar cod experiments resulted in similar target strength values with a mean target strength of -41 dB re 1 m$^2$ for the 94-158 kHz bandwidth for the first polar cod experiment with the slightly larger individuals and -42 dB re 1 m$^2$ for the 94-158 kHz bandwidth for the smaller polar cod experiment. In the 189-249 kHz bandwidth, both polar cod experiments resulted in an average target strength of -44 dB re 1 m$^2$. The northern shrimp had the weakest echo intensity with a mean target strength of -78 dB re 1 m$^2$ and -82 re 1 m$^2$ in the 94-158 kHz and 189-249 kHz bandwidth, respectively. All species had a mean target strength that decreased in the higher frequency range. \\
Atlantic cod had the largest variability in target strength intensity per individual (i.e., among pings forming a track) with a maximum range of 43 dB re 1 m$^2$ at the nominal frequency, 120 kHz, and 33 dB re 1 m$^2$ at the nominal frequency, 200 kHz. The variability in target strength intensity per track at the nominal frequency for the polar cod and northern shrimps were largest at 200 kHz but smaller than the Atlantic cod target strength intensity variability. During the second polar cod experiment, the polar cod had a maximum target strength intensity range within a track of 21 dB re 1 m$^2$ at 200 kHz, and for northern shrimp it was 8 dB re 1 m$^2$ at 200 kHz.\\


\subsection{Classifier training}
The three classifiers trained on the normalised target spectra (Figure~\ref{fig:FigureAFKABAN4}E-H) showed a high performance in classifying the frequency response of polar cod, Atlantic cod, and northern shrimp across both the 94-158 kHz and 189-249 kHz bandwidths (mean class-weighted F1 scores: >95\%; Tables~\ref{tab:F1_AFKABAN}). The northern shrimp target spectra had the highest mean per-class classification performance in the 94-158 kHz bandwidth for all three classifiers ($\geq$98\%; Table~\ref{tab:F1_AFKABAN}). Atlantic cod had a slightly higher performance than polar cod (up to 0.03 increase) in both bandwidths. Both complex and computationally expensive classifiers, LightGBM and SVM, did not have notably higher performance than kNN, and the kNN classifier required at least 10x less computing time to train. \\

\begin{muntab}{|l|c|c|c|c|c|c|}{F1_AFKABAN}{Classifier F1 scores estimated by nested cross-validation (mean ± SD) of the normalized target spectra collected with the 120 kHz and 200 kHz transducer.}
\hline
& \multicolumn{3}{c|}{120 kHz} & \multicolumn{3}{c|}{200 kHz} \\
\cline{2-7}
& kNN & LightGBM & SVM & kNN & LightGBM & SVM \\
\hline
Mean  & & & & & & \\
class-weighted & 0.96 $\pm$ 0.01 & 0.97 $\pm$ 0.01 & 0.97 $\pm$ 0.01 & 0.97 $\pm$ 0.01 & 0.97 $\pm$ 0.01 & 0.97 $\pm$ 0.02 \\
\hline
Atlantic cod & 0.95 $\pm$ 0.02 & 0.98 $\pm$ 0.01 & 0.96 $\pm$ 0.02 & 0.98 $\pm$ 0.01 & 0.96 $\pm$ 0.02 & 0.96 $\pm$ 0.04 \\
\hline
Polar cod & 0.94 $\pm$ 0.02 & 0.95 $\pm$ 0.02 & 0.95 $\pm$ 0.02 & 0.96 $\pm$ 0.01 & 0.95 $\pm$ 0.03 & 0.96 $\pm$ 0.03 \\
\hline
Northern  & & & & & & \\
shrimp & 0.98 $\pm$ 0.01 & 0.97 $\pm$ 0.01 & 0.99 $\pm$ 0.01 & 0.98 $\pm$ 0.0 & 0.99 $\pm$ 0.01 & 1.0 $\pm$ 0.0 \\
\hline
\end{muntab}

\section{Discussion}
\subsection{Species-specific patterns}
The high classification performance (mean class-weighted F1 score of 97\%) a relatively computationally inexpensive machine learning algorithm for three coincident species with a single transducer and is promising for \textit{in situ} classification of targets from broadband echosounders on ships or uncrewed platforms. The results show that three coincident species, Atlantic cod, polar cod, and northern shrimp, can be differentiated using their target spectra. Presumably, the range of target spectra complexity and morphological differences of the three species ensured the high performance of the classifiers. \\
Atlantic cod's target spectra were found to be the most complex, defined by closer and more intense peaks and null across the bandwidth. The spectral complexity observed in the Atlantic cod target spectra could have suggested that the SED contained interference from other targets (Figure~\ref{fig:FigureAFKABAN3}G; \citealt{Khodabandeloo2021, Stanton1996}). However, the rigorous manual target selection process ensured that only one individual was included per SED and no adjacent targets were included in the Fourier transform window ($\sim$0.25 m above and below the target). Therefore, the multiple scattering features within the individual Atlantic cod targets must have originated from the backscatter of different organs interfering with each other and causing the response signal to be affected by constructive and destructive interferences \citep{Demer2017, Reeder2004}. We thus expect that discriminating and classifying several morphologically complex targets, such as Atlantic cod, will be more challenging \citep{Au2003, Clay1991, Clay1992}. \\
In contrast, polar cod target spectra had an intermediate complexity with some ripples and a relatively consistent slope across the spectra. During the target selection of polar cod, there was only one central dominant SED per ping, which suggested each individual had a single dominant scattering feature (i.e., the swimbladder) and explained the absence of large nulls and peaks (Figure~\ref{fig:FigureAFKABAN3}H). The northern shrimp had a mixed spectral complexity in the target spectra with some ripples in the 94-158 kHz bandwidth but predominantly flat normalised target spectra, especially in the 189-249 kHz bandwidth (Figure~\ref{fig:FigureAFKABAN4}D). The emitted chirp from the 120 kHz transducer had a 10 kHz wider bandwidth than the 200 kHz transducer, which increased the temporal resolution to 9 mm (compared to 11 mm for the 200 kHz). The finer temporal resolution from the wider bandwidth may have revealed finer-scale scattering features, which are typically only discernable with higher frequencies \citep{Reeder2004}.\\
Target spectra complexity was used by \citet{Cotter2021} to scrutinise target spectra into four classes based on selected scattering models (i.e., above, at, or below resonance for gas-bearing organisms or fluid-like organisms). These categories were used to classify mesopelagic fish into size classes with a mean F1 score of 0.90. Similarly, \citet{Roa2022} had a high performance (best mean class-weighted F1 score was 0.96) with classifiers trained on scattering models for six different reef fish. They found that the nodes or ``ripples," typically found at higher frequencies, were the prominent source of discriminating information. Discriminating nodes and ripples were not found in three of the four modelled zooplankton groups in \citealt{DunnInPrep} submitted to this volume, which resulted in moderate performance for the classifiers (best mean class-weighted F1 score was 0.71). Based on previous studies and the results from this study, we conclude that classifying targets with different spectral complexity can positively impact classification performance. 


\subsection{Intensity variability of broadband target spectra}
Broadband acoustic backscattering signals exhibit large unexplained variability between detections of a single target \citep{Reeder2004, Gugele2021, Dunning2023}. For example, an Atlantic cod target spectra study recorded a maximum target strength variation of 30 dB re 1 m$^2$ within a track of a single fish at 38 kHz \citep{Dunning2023}. Here, we observed a comparable maximum variation in target strength of 33 dB re 1m2 at 200 kHz with an Atlantic cod track. However, polar cod and northern shrimp exhibited a smaller variation of target strength per track. The target strength variability in broadband acoustics for a single target was found to be greater than could be explained with tilt angle or fish length \citep{Dunning2023}, which are traditionally used to explain the variability in narrowband target strength measurements \citep{Khodabandeloo2021, Zhang2021}. Presumably, the stochasticity found in the Atlantic cod target spectra tracks could be due to variations in the section of the fish being ensonified from ping to ping. A mesocosm experiment, similar to this study but with fewer individuals and optical verification, could develop a better understanding of broadband acoustic target strength variability. \\
In the classification, the normalising preprocessing algorithm removes the intensity component of the target spectra (Figure~\ref{fig:FigureAFKABAN4}I-L). Normalising the target spectra had the largest effect on the intensity variability of northern shrimp spectra. Though northern shrimp had the smallest maximum variability per track, 7 dB re 1 m$^2$ at 120 kHz and 8 dB re 1 m$^2$ at 200 kHz, the intensity between individuals varied greatly, especially over the 189-249 kHz bandwidth (Figure~\ref{fig:FigureAFKABAN4}D). The normalised shrimp target spectra reduced variance, which showed that the northern shrimp had the most consistent target spectra pattern despite the large variability in target strength intensity.


\subsection{Classification of \textit{in situ} targets}
The high classification performance of the classifiers in a controlled experiment is an important step towards in situ target classification. However, fundamental challenges should be addressed before in situ target classification can be achieved with mesocosm-trained classifiers. A significant limitation of supervised classification is the dependence on collecting training datasets for all classes \citep{Handegard2021}. Collecting target spectra from mesocosm experiments for all species and size groups in complex and dynamic environments such as the ocean, even in Arctic regions with relatively low species diversity, is unrealistic. A series of ship-based downward-looking lowered acoustic probe experiments were completed as part of this study, attempting to classify in situ targets using the trained classifiers. However, the trawls showed the community was dominated by herring and capelin among the Atlantic cod, polar cod and northern shrimps, which prevented validation of \textit{in situ} classification. One method to validate the classifiers would be to repeat the lowered acoustic probe experiments in an enclosed region, such as a lake or smaller fjord, dominated by a single species to assess the error for that class. Single species-dominated regions are commonly used in fisheries acoustics to associate the backscatter to a single species (e.g., \citealt{Geoffroy2016, DeRobertis2019}. A more widespread method to use mesocosm-trained classifiers would be to have broader classes and to group species based on morphological features and expected backscattering \citep{Gugele2021}. However, better knowledge of the impact of multiple scattering features and their contribution to target spectra complexity will also be necessary to successfully classify \textit{in situ} broadband acoustic signals.\\
 Another practical limitation to \textit{in situ} broadband acoustic target classification is the rigorous track selection requirements. Better tracking algorithms for broadband data with reduced risk of interferences from contaminating targets within the Fourier transform window will need to be developed. Currently, tracks tend to be manually selected in broadband acoustics studies \citep{Khodabandeloo2021, Cotter2021, Dunning2023}, which is time-consuming and subjective. Manual selection of single echoes and tracks halts the potential of automation and reproducibility. With automatic and reproducible track selections, classifiers could be quickly applied to new datasets for large-scale analysis of hydroacoustic survey datasets \citep{Chawarski}.\\

\section{Conclusion}
Three coincident species (Atlantic cod, polar cod, and northern shrimp) were found to have distinct enough target spectra relative to each other, despite their intraspecies variability, to be discriminated at a high performance by machine learning classifiers' success was due to the different levels of target spectra complexity observed across the selected species. Further mesocosm studies will determine the taxonomic resolution to which mesocosm-trained classifiers can be used for \textit{in situ} classification, either by adding new classes of additional coinciding species, such as herring and capelin, or by joining new classes in the existing ones based on their target spectra complexity. This study paves the way toward automating \textit{in situ} species classification using lowered acoustic probes or autonomous underwater vehicles equipped with broadband echosounders, which opens the possibility to real-time warnings of bycatch risks  to reduce  cost and trawling impact. Forecasting bycatch risks could greatly impact the shrimp fishery because excessive retention of non-regulated bycatch can increase fuel costs, loss of revenue, and practical problems of onboard with sorting the catch \citep{Jacques2022}. Finally, such automating classification methods could increase the understanding of ecosystem changes happening as a result of environmental changes in the North Atlantic and the Barents Sea and modifying ecological interactions between Arctic fish species through distribution shifts \citep{Fossheim2015, Morley2018, Morato2020}.


