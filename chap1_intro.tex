\chapter{Introduction}
\label{chap:intro}

\textit{ Introduction or Overview section that provides all of the following information:
•	 a comprehensive review of relevant literature
•	 how your research fits into the larger context of your field(s) or discipline(s)
•	 the objectives of your thesis research
•	 a statement that makes clear the coherence of the chapters to follow}


\section{General Introduction}


\munepsfig[scale=0.8]{Figure3_TS}{Target strength of marine animals at a range of frequencies. Swimbladdered fish is representative of a 20 cm Atlantic cod (\textit{Gadus morhua}), Euphausiid is representative of a 25 mm \textit{Thyssanoessa inermis}, Hydrozoan is representative of 15 cm \textit{Aglantha digitale}, Pteropod is representative of a 1.5 mm \textit{Limacina retroversa} and Copepod is representative of a 5 mm \textit{Calanus finmarchicus}. The vertical grey dashed lines indicate commonly used frequencies in fisheries acoustics.}



\section{Chapter outline and research objectives}
This thesis in structured in 5 chapters. Chapter 1 provides an introduction to the main themes discussed throughout the thesis, Chapter 2, 3 and 4 are research papers and Chapter 6 provides the final conclusions.\\
In Chapter 2, I conducted a comparative study of the zooplankton and ichthyoplankton density estimates in near-surface sound scattering layers using four different methods. Two of the methods were by direct sampling from a research vessel (mesozooplankton net (MultiNet), macrozooplankton trawl (Tucker trawl)) and the other two methods used broadband acoustic data collected from an autonomous surface vehicle (forward and inverse methods). I compared the density estimates results and contextualized the results in terms of each method's expected biases and calculated uncertainties. I discussed the importance of new solutions for surveying ecosystems with emerging technologies.\\
In Chapter 3, I examined the potential of increasing the taxonomic resolution of acoustic surveys with autonomous platforms by classifying the target spectra of zooplankton. I trained three conceptually different supervised learning classification algorithms with modelled target spectra of four different Arctic zooplankton groups. I validated the classification predictions against observations collected in a mesocosm of a known mixed zooplankton community. I discussed the limitations of the tested method and provided recommendation for model-informed classification of zooplankton.\\
In Chapter 4, we investigated the potential of discriminating between coincident species in the Arctic using only their measured target spectra. we conducted single-species mesocosm experiments to collect the target spectra of free swimming Atlantic cod, polar cod and northern shrimp. The target spectra measurements were used to train three machine learning classification algorithms. we discussed the feasibility of expanding the supervised classification of mesocosm-informed classification for \textit{in situ} measurements of coincident species from a lowered acoustic probe or diving autonomous platform. The publication resulting from this chapter's research was co-led by Chelsey McGowan-Yallop.\\
In Chapter 6, I summarized the results and contributions of the research from Chapters 2, 3 and 4. I discussed limited availability of ecosystem monitoring data in the Arctic and the use of autonomous platforms equipped with broadband echosounders as a tool to increase the monitoring potential and ecosystem understanding in the Arctic and beyond. I propose directions for future research to incorporate acoustic measurements in ecosystem research.//
