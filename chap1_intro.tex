\chapter{Introduction}
\label{chap:intro}

\section{Context for thesis}
The Arctic is rapidly changing due to physical environmental changes driven by climate change. The rapid changes (e.g., early ice breakup and reduced coverage (Stroeve et al., 2012) and increased flow of warm Atlantic waters (Wang et al., 2020), and changing sea temperatures (Steele et al., 2008) are increasing both the industry operations development (commercial fishing, oil and gas exploration, and shipping) and the ecosystem changes (Fossheim et al., 2017; Frainer et al., 2017; Doney et al., 2012) in the Arctic.
Sustainable industry development in the Arctic requires better ecosystem understanding and increased environmental monitoring. However, environmental monitoring in the Arctic can be challenging. The harsh and remote areas of the Arctic present logistical and financial barriers to frequent ship-based surveys in the Arctic. Furthermore, the Arctic has a strong seasonal cycle which alternates between polar night in the winter and midnight sun, with constant sunlight, in the summer. Until recently, the polar night, or winter, was not of interest for ecological studies because it was assumed that it was a time of biological quiescence. However, ecological processes remain high during this dark period (Berge et al., 2015 unexpected; Ludvigsen et al., 2009). We still have a lack of understanding of the polar night, and therefore, the full seasonal cycle of the Arctic marine ecosystem should be monitored to understand the impacts of industry developments in the Arctic (Berge et al., 2015, review). \\

Figure 1 :Randall photo of ship?

Monitoring undistrupted pelagic species with traditional methods is already a difficult task in daylight because of the net and trawl sampling biases (Skjoldal et al., 2013) but in the dark, the artificial light from the ship further impacts the behaviour of pelagic species down to at least 200 m (Berge et al., 2020 artifical). Hydroacoustics surveys are a monitoring method that does not require lights from large vessels and can be used to completement the traditional net and trawl surveys. Acoustics is a non-invasive monitoring method that can be mounted on a mooring or buoy to study an undisrupted pelagic ecosystem. Acoustics systems can collect long time series, which are particularly valuable in Arctic where it is expensive and logistically challenging to collect baseline measurements.\\

The Arctic is not the only place undergoing significant ecosystem changes with limited baseline information. Globally, and in Canada, fisheries management has minimal consideration for ecosystem processes and is predominantly single-species focused (Skern et al., 2016, Pepin et al., 2020). Nevertheless, there is a widespread agreement that, to harvest the aquatic environment sustainably, there needs to be a shift towards a more holistic approach to fisheries management (Hilborn et al., 2004, Fao et al., 2003, Brodziak et al., 2002). This shift is called the ecosystem-based approach to fisheries management. It requires consideration of the impacts of the fishery on habitat, predator and prey interactions, as well as social impacts (Link et al., 2011). It also includes the increasing recognition of the effects of climate, weather, environmental conditions, and food web dynamics on the targeted stock (Fernandino et al., 2018, Tam et al., 2017). Therefore, comprehensive monitoring of living marine resources is fundamental to a successful ecosystem-based approach to fisheries management. In fact, Canada's Oceans Act states that to perform it's duties and function, Fisheries and Oceans Canada may "conduct marine scientific surveys relating to fisheries resources and their supporting habitat and ecosystems" and "participate in ocean technology development" (Canada, 1996). Ecosystem-based fisheries management requires baseline data on interspecific interactions and their connections with environmental factors, which will involve much more data inputs and technologies than traditional ship-based methods can provide (Van et al., 2013). With increasing fishing efforts combined with dramatically declining fish species stocks, good ecosystem-based fisheries management is urgently needed (Aronica et al., 2019, Yassir et al., 2023).\\
In recent years, there has been significant technology and scientific development in three fields that are increasing the ecosystem monitoring potential, which are particularly valuable for addressing the challenges of ecosystem monitoring the Arctic. These developments are: 1) commercial availability of broadband echosounders, 2) autonomous ocean monitoring platforms;  and 3) development and access to software for sound scattering models and machine learning algorithms. In this thesis, I make use of these recent technology and scientific breakthroughs towards improved methods for monitoring and understanding Arctic ecosystems.\\

\section{Acoustic monitoring}
\subsection{Active acoustics}
Monitoring marine resources is an important part of sustainable fisheries and ecosystem services. Traditional large-scale scientific trawl surveys (Chadwick et al., 2007) are routinely used for stock assessments for fisheries management bodies. These extractive surveys can be complemented with active acoustics, or hydroacoustics, to increase the spatial resolution of the station-based sampling from the trawls for pelagic species (Mowbray, 2014). Active acoustic is a powerful tool for aquatic monitoring because sound travels much further underwater than light. Furthermore, the echo of a reflected sound wave contains information about the fish or object that reflected the sound therefore, it is used for remote observations and monitoring of aquatic species (Simmonds and MacLennan, 2005). It differs from passive acoustics, which records sounds emitted by other sources, such as whale songs. \\

Hydroacoustics surveys use echosounders, which consist of a transceiver and one or many transducers. The transducer converts the electrical energy sent from the transceiver into mechanical energy, an acoustic pulse (Urick, 1983; Simmonds and MacLennan, 2005). The transducer is typically mounted on a ship's hull, submerged in the water and pointing downwards. The acoustic pulse travels through the water column and is reflected by surfaces and objects with acoustic impedance that differs from that of the surrounding medium, seawater or fresh water. The impedance is the product of the acoustic density (rho) of the object and the speed of sound (c) (z=rho*c) (Medwin and Clay, 1998). Then, the transducer takes on a listening role, records the reflection of the emitted acoustic pulse (the echo), amplifies, and converts the echo into an electrical signal to send to the transceiver. The received acoustic pulse contains information on the impedance of scatterers and backscatter throughout the water column. The temporal delay of the backscatter informs on the distance between the target and the transducer face. The radial location of the target in the acoustic beam can be calculated with a split-beam transducer. The split-beam transducer records the backscatter through three or four quadrants; the phase difference between the signal from each quadrant can be used to locate the target in the beam (Ehrenberg and Torkelson, 1996). Split-beam technology has had an important impact on fisheries acoustics because the backscatter can compensate for the beam pattern by determining the target's location. The acoustic beam pattern is conical, much like a flashlight beam, with most of the energy concentrated in the centre. \\

 
Figure 2: Image of a wideband autonomous transceiver (yellow cylinder ; Kongsberg, Horten, Norway) with a 38 kHz split-beam transducer (orange). Photo taken by Stig Falk-Petersen.
\subsection{Quantifying acoustic backscatter}
The backscattered energy measured by the transducer during hydroacoustic surveys is typically quantified by the fraction of incident energy scattered back to the transducer, described as the backscattering cross-section, $\sigma_{bs}$:
\begin{muneqn}{sigbs}
\sigma_{bs}=R^{2}\frac{I_b}{I_i }
\end{muneqn}


where $I_b$ is the backscattered intensity, $I_i$ intensity of the incident wave, and R is the range from the target where $I_b$ and $I_i$ are measured. In fisheries acoustics, we often use logarithmic scale, in decibels, because of the wide scale of sizes for aquatic organisms, from zooplankton to fish (Simmonds and MacLennan, 2005). The backscattering cross section is then commonly described as Target Strength (TS):
\begin{muneqn}{TS}
TS=10\log_{10}\sigma_{bs},
\end{muneqn}

Typically, TS is used to describe the ability a single target (fish or zooplankton) to reflect sound. However, when considering an aggregation or layer of single targets that are is too dense to resolve individuals, it is more appropriate to use the volume backscattering coefficient, sv:
\begin{muneqn}{sv}
s_v=\frac{\sum_{i=1}^{N}\sigma_{bs}^{i}}{V}
\end{muneqn}

Where N is the number of individual int the volume, $\sigma_{bs}^{i}$ is the cross-section of each individual, V is the sampling volume of the acoustic pulse (MacLennan and Simmonds, 2013). The sampling volume can be related to the acoustic beam pattern as:
\begin{muneqn}{samplingV}
V =\frac{c\tau\psi R^2}{2}  
\end{muneqn}

where the range is R, the pulse length is $\tau$, and the sound speed velocity is c. The sample volume represents a measure of the beam width (MacLennan and Simmonds, 2013) and results in the direction of the target relative to the origin of the transducer. The equivalent beam angle, $\psi$, indicates the solid angle of an idealized acoustic beam:
\begin{muneqn}{EBA}
\psi =\int^{\pi}_{\theta=0}\int^{2\pi}_{\phi=0}b4(\theta, \phi)sin(\theta) d\theta d\phi,
\end{muneqn}

where for an ideal cylindrical transducer, the directivity, or beam pattern, is expressed as:
\begin{muneqn}{bessel}
b\ =\frac{2J_1(ka\ sin(\theta))}{ka\ sin(\theta)},
\end{muneqn}
Where $k$ is the wavenumber, $a$ is the transducer radius, and $J_1$ is a Bessel function of the first kind (Medwin and Clay, 1997). Similarly to the backscattering cross section ($\sigma_{bs}$), the volume backscattering coefficient is commonly expressed on a logarithmic scale as volume backscatter, $S_v$:
\begin{muneqn}{Sv}
S_v=10\log_{10}s_v.
\end{muneqn}

The measure of $s_v$ summarizes the aggregation inside the acoustic sampling volume because of the linearity principle of fisheries acoustics \citep{Foote1983}, which is expressed as:
\begin{muneqn}{av}
A_v\ \varepsilon\ =\ n_o<G \ b^2\ \sigma_{bs}\ >,
\end{muneqn}
where $A_v$ $\varepsilon$ is the mean echo energy of the impinged target, $n_o$ is the number density, and $<G\ b^2\ \sigma_{bs}\ >$ is the ensemble average of the distribution of the characteristics of the targets; G is the gain factor, b2 is the product of transit and received beam pattern and $\sigma_{bs}$ backscatter cross-section of the targets \citep{Foote1983}. Equation~\ref{eq:av}) states that the echo energy from a volume containing a random distribution of scatterers is on average equal to the sum of scattered echo energy from each individual within the volume (Benoit-Bird, 2009, Greenlaw, 1979).

\subsection{Hydroacoustic surveys}
Based on these fundamental fisheries acoustics principles, hydroacoustic surveys commonly use narrowband echosounders (acoustic pulse containing a single frequency) to extend the spatial resolution of trawl data (Parker-Stetter et al., 2009) and calculate biomass. The theoretical TS values of the species found in the trawl are calculated using sound scattering models, they are summed based on their relative abundance and scaled to estimate the absolute abundance corresponding to the measured $S_v$. The measured Sv is often integrated between depth layers to get a measure of the nautical area backscattering coefficient (NASC or $s_A$) for a larger area:
\begin{muneqn}{NASC}
NASC = 4\pi \ {1852}^2 \ {10}^{Sv/10} \ T,  
\end{muneqn}

where the 4$\pi$ is remnants from historic uses of spherical scattering coefficient ($4\pi\sigma_{bs}$, assumes omnidirectional scattering) , the integer, 1852, is the conversion for units from meter to nautical mile, and T is the thickness of the integrated layer (MacLennan et al., 2002). For large-scale surveys, NASC is a common measure (Parker-Stetter et al., 2009; MacLennan et al., 2002). However, as demonstrated by the linearity principle, many small weak scattering targets can have the equivalent volume backscatter to fewer targets with a higher impedance. Therefore, narrowband acoustic surveys rely on trawling for species identification and length composition (De Robertis et al., 2021). Hydroacoustic surveys typically do not increase species richness or biodiversity knowledge, but they can inform vertical migrations and predator-prey interactions (Skaret et al., 2020; Macaulay et al., 1995). Furthermore, a species' life history can be used to target a specific age group of a species with pelagic acoustic surveys. For example, many species are pelagic in their juvenile stage, such as Atlantic cod (Gadus morhua) and polar cod (Boreogadus saida), and can be detected with pelagic acoustic surveys (Bouchard et al., 2017; Nielsen and Lundgren, 1999). In regions dominated by a single species, narrowband hydroacoustic surveys can be used without coincident additional evidence (i.e., trawling) because all the backscatter can be attributed to a single species (De Robertis et al., 2021; Reiss et al., 2021; Geoffroy et al., 2011). However, hydroacoustic surveys typically depend on knowledge of species composition, body size and density data to translate active sonar signals to abundance or biomass (McClatchie et al., 2000; Fernandes et al., 2016).\\
To reduce the dependence of trawling, different discrete narrowband frequencies at wide intervals can be used to isolate the backscatter contribution from targets or the volume backscatter of different classes of targets (Figure 3). This method is called multifrequency analysis or frequency differencing (Korneliussem et al., 2018). For example, a common technology used for vertical migrations and predator-prey interactions is the Acoustic Zooplankton Fish Profiler (AZFP) because it has a transducer available for different narrowband frequencies (38 to 2000 kHz). The wide range of available frequencies enables the detection of fish (typically detected with lower frequencies <=200 kHz) and zooplankton (typically detected with higher frequencies >=200 kHz) with a single instrument (Simmonds and MacLennan, 2005, p.66). AZFPs are commonly installed on moorings for low spatial coverage but high temporal resolution because of their long-term sampling capabilities. Recently, an AZFP on a bottom mooring has been used to study diel vertical migrations by fish schools to predict the overlap with tidal stream renewable energy devices (Whitton et al., 2020). \\
 

\munepsfig[scale=0.8]{Figure3_TS}{Target strength of marine animals at a range of frequencies. Swimbladdered fish is representative of a 20 cm Atlantic cod (\textit{Gadus morhua}), Euphausiid is representative of a 25 mm \textit{Thyssanoessa inermis}, Hydrozoan is representative of 15 cm \textit{Aglantha digitale}, Pteropod is representative of a 1.5 mm \textit{Limacina retroversa} and Copepod is representative of a 5 mm \textit{Calanus finmarchicus}. The vertical grey dashed lines indicate commonly used frequencies in fisheries acoustics.}

\subsection{Broadband  echosounders}
Building on the benefits found with multifrequency analysis, acoustic remote sensing technology has advanced from narrowband to broadband echosounders. The wider bandwidth made available by broadband echosounders returns backscatter measurements across a wider range of frequencies, offering discrimination and characterization of targets (i.e. fish or zooplankton) (Lavery et al., 2017, Bassett et al., 2020, Benoit-Bird and Waluk, 2020) (Figure~\ref{fig:Figure3_TS}). The broadband acoustic pulse is typically referred to as a frequency-modulated chirp, with the frequency increasing linearly throughout the acoustic pulse. Broadband echosounders improve the range resolution and signal-to-noise ratio relative to narrowband echosounders. These improvements result from the single processing technique of pulse-compression or matched filter. The matched filter output, yR(t) is calculated by:
\begin{muneqn}{matchfilter}
{\ y}_R(t)\ =\frac{\ {\ v}_R(t)\ \otimes\ {\ v\ast}_T(t)\ }{{|v_T(t)|}^2},
\end{muneqn}
where vr is the received pulse, vt is the transmitted pulse, ⊗ is the cross-correlation and * is the complex conjugate (Andersen et al., 2020:not peer reviewed; Loranger et al., 2023). Through the cross-correlation, the matched filter systematically compares the received signal with the pattern of the emitted signal. Stochastic noise does not contain the pattern of the emitted signal, and therefore the match filter results in a signal with dampened noise. Whereas a discrete target, such as a fish, will reflect a mirror image of the emitted pulse and result in a narrow peak in the matched filtered signal. 
Commercial broadband echosounders are relatively new (~2011), and with new commercially used technology comes new artefacts and problems to be resolved and potentially modified data processing conventions. For example, the beam pattern (equation x, b) varies with frequency through the wavenumber (k=f/c). As a result, the sampling volume is reduced throughout a single frequency modulated up-chirp (i.e., the frequency increasing linearly throughout the pulse duration), which is expressed as a positive trend in volume backscatter with frequency (Eq. # Sv, V) (Urmy et al., 2023, Medwin and Clay, 1998). Furthermore, the datasets are ~10x larger than narrowband datasets. Narrowband data processing methods that require expert scrutiny and visual assessments do not transfer well to broadband datasets because of the size of the files. Therefore, smaller subsamples must be analyzed with more powerful algorithms. 
Autonomous ocean monitoring platforms
Until recently, oceanography was a data-limited field with data collection solely dependent on expensive research vessels for large survey campaigns. Recent scientific and technological advances are moving physical, chemical and biological oceanography to data-rich fields (Malde et al., 2020). Emerging technologies, in particular autonomous or uncrewed, ocean monoring plarforms, stem from these scientific and technological advances and can provide a new perspective to regions that have traditionally been surveyed with large research vessels. Autonomous ocean monitoring vehicles can increase the spatial extent, temporal resolution, and taxonomic resolution of environmental monitoring and provide measurements for biophysical assessments (Greene et al., 2014). Indeed, many scientific and technological advancements for ecosystem monitoring with autonomous platforms are non-lethal and have minimal impact on the ecosystem (Trenkel et al., 2019). 
A prominent benefit to autonomous or uncrewed ocean monitoring platforms is the reduced disturbance from light and noise. Fish within the epipelagic layer (0 - 100 m) react to light from vessels (Ludvigsen et al., 2018) and vessel noise, even when using noise-reducing state-of-the-art research vessels (Ona et al., 2007). Therefore, research vessel surveys report deeper and fewer fish detections for shallowly distributed fish than autonomous platform surveys (De Robertis et al., 2019). 
An additional benefit to autonomous platforms is that they range in size, speed, endurance, depth coverage (Benoit-Bird et al., 2018) and sensor capacity. Diving platforms, such as gliders, can collect measurements at depths for extended periods of time (Benoit-Bird et al., 2018), whereas surface platforms can collect undisturbed near-surface data. Typically, hydroacoustic surveys have an acoustic blind zone, the depth at which data collection begins, that can extend ~15 m below the surface with a hull-mounted transducer from a research vessel (Scalabrin et al., 2009), but autonomous or uncrewed surface platforms tend to be much smaller and have a shallower hull, which reduces the acoustic blind zone to < 5m.
Despite the clear benefits of autonomous platforms, there remains resistance to changing the status quo because of the complexities involved in integrating new data streams (Wilson et al., 2011, Fujita et al., 2021). For example, data analysis pipelines for autonomous hydroacoustics surveys have not maintained the same pace as technology advances, causing a bottleneck and a delay in transferring information to end-users of the data (managers and policymakers) (Malde et al., 2020). Autonomous platforms equipped with echosounders can be used in areas dominated by a single species because all the backscatter can be estimated from the single dominant species (De Robertis et al., 2019; Bandara et al., 2022). De Robertis et al. (2021) present the first fully autonomous acoustic fisheries survey for stock assessment of walleye pollock (Gadus chalcogrammus) without trawling in 2020. Younger pollock, aged 2-4, are more pelagic and can be attributed to the "pre-recruit" biomass for abundance at age indices for fisheries management (De Robertis et al., 2021).
Since the ocean is typically dynamic and contains diverse species assemblages, most areas require data analysis methods to be extended to species mixtures. Broadband echosounders are a promising tool for species identification in species mixtures because they can be leveraged to extract a wider backscatter spectrum from a target and can be used to extract identifying features for identification (Urmy et al., 2023; Cotter et al., 2021, class..; Benoit-Bird and Waluk, 2020).  
\section{Scattering models}
Technology and computation developments have also improved numerical and analytical approaches to sound scattering models. Sound scattering models are used to predict the acoustic backscatter of a target (typically fish or zooplankton) (Jech et al., 2015). The target shape, orientation, and material properties are key parameters of sound scattering models. However, these can vary within a population and can be difficult to measure in situ (Sakinan et al., 2019, Smith et al., 2010). We can rely on literature values for study regions with comparable water masses and environmental conditions. Several sound scattering models are available depending on the type of acoustic target; each model has limitations and advantages. A summary of available models is published in Jech et al., 2015. The following is a summary of two commonly used scattering models:
\subsection{Distorted Wave Born Approximation}
The Distorted Wave Born Approximation (DWBA) is mainly applied to weak scatterers that have material properties that are similar to water (plankton; Stanton et al, 1993, 1996, Stanton and Chu, 2000, or non-swimbladdered fish; Gorska et al., 2005). There are a few variants of the DWBA. For example, the stochastic variant (SDWBA) which is commonly used for Antarctic krill (Euphausia superba) to account for the stochastic nature of the scattering as there results of body curvature changes while swimming (Calise et al., 2011, Demer and Conti 2003). Another variant is the phase-compensated version (PC-DWBA), which accounts for the scattering-induced attenuation due to densely aggregated zooplankton (Chu and Ye, 1999) (Figure 3, euphausiid, hydrozoan and copepod). Overall, the advantages of the DWBA and its variants are the flexibility to scattering geometry, orientation and acoustic frequency (Jech et al., 2015). The main limitation is that it is only applicable to fluid-like scatterers. 
Viscous-elastic model
Feuillade and Nero (1998) developed the viscous-elastic scattering model to include the scattering of the swimbladder wall, surrounding flesh and the gas enclosed. The outer shell, typically fish flesh, is the viscous layer and the elastic shell, swimbladder wall, is the elastic layer. Together, they affect the resonance of the swimbladder and its backscatter. Khodabandeloo et al. (2021) applied the model to mesopelagic fish and compared it with in situ measurements. The advantage of the viscous elastic model is that it includes the higher modes of scattering, which is particularly important for higher frequencies (Khodabandeloo et al., 2021) (Figure 3, swimbladdered fish and pteropod). As implemented in Khodabandeloo et al. (2021), a prominent limitation of this method is the assumption that the gas enclosure is spherical. The simple sphere shape was chosen to reduce the computational expense, but it ignores the realistic aspect ratios of the swimbladder, which tend to have the shape of a prolate spheroid (Khodabandeloo et al., 2021).
For all models, assumptions have to be made for morphological and material properties parameters, which can affect the shape and amplitude of the results. Sound scattering models can be run as ensembles to capture the study region's variability in shape, orientation, and material properties. Model ensembles repeat calculations with a random selection of parameters within the given parameter distributions. Model ensembles can be particularly valuable for averaging over orientation for volume backscatter inversions (Amakasu et al., 2017; Stanton et al., 1993 average). 
Machine learning in fisheries acoustics
From face recognition to self-driving cars, artificial intelligence is increasingly being applied to datasets of all types. Machine learning, a subfield of artificial intelligence, offers algorithms that can be used to solve problems in a range of domains, including fisheries acoustics. Machine learning supports data-driven learning and results in automated decision-making (Beyan and Browman, 2020), thus potentially reducing human review effort and user subjective bias from visual assessments during data analysis. In particular, machine learning algorithms are practical for sensors and platforms that collect large amounts of data, like autonomous platforms and broadband echosounders. With machine learning, patterns and mathematical expressions are found in past data to make judgements about new data (Nguyen and Armitage, 2008). 
Machine learning can be categorized into four learning methods: supervised learning, unsupervised learning, semi-supervised learning and reinforcement learning (Zhao et al., 2021). Supervised learning is the most common type of learning, including in fisheries acoustics. With supervised learning, predictions are made on new data through continuous learning from training data based on predictor features (Kotsiantism, 2007). Classification and regression type of problems are commonly solved with supervised learning (Zhao et al., 2021). Example uses of supervised learning in fisheries acoustics are predicting the dominant species of an aggregation or school by using multifrequency Sv of acoustic fish school and other fish school descriptions (morphological, bathymetric and positional) (e.g., Fallon, 2016; Fernandes, 2009) or, more recently, classifying species using modelled target strength spectra (Cotter et al., 2021; Roa et al., 2022). Unsupervised learning finds patterns and representations in the data without the requirement of labelled training data (Yassir et al., 2023). Unsupervised learning is predominantly used for clustering or dimensionality reduction. Unsupervised learning may be preferred in fisheries acoustics for studies with minimal supporting biological information (Agersted et al., 2021). For example, to differentiate between scattering layers based on volume backscatter spectra (Sv(f)) (Ross et al., 2013), to cluster mesopelagic targets based on their target spectra (TS(f)) (Agersted et al., 2021) or to optimize parameters in regression TS to length models (Stevens et al., 2021).
Semi-supervised learning combines supervised and unsupervised learning, where labelled and unlabelled datasets are used to realize a combination of classification, clustering and regression (Zhao et al., 2021). Semi-supervised classification was used by Choi et al. (2021) to delineate sandeel schools in Sv measurements by clustering and classification trained by labelled and unlabelled data. The semi-supervised method required only 10% of the training data to be labelled, thus reducing the dependency on user expert knowledge and visual assessments (Choi et al., 2022). Meanwhile, reinforcement learning is used for autopilot and uncrewed operations because it is a complex ML method that constantly interacts with the outside world (i.e., new information) (Montague, 1999, Zhao et al., 2021). It is not commonly directly used for fisheries acoustics, but it is used for uncrewed marine platforms. For example, reinforcement learning has been used for obstacle detection and avoidance (Cheng and Zhang, 2018).  
A particular limitation of machine learning is feature engineering. Feature engineering is the selection, manipulation and transformation of the raw input data into new variables for the ML algorithms. These are predominantly the steps in machine learning where data processing workflow continues to require user manipulation and decisions. Deep learning methods automate the feature engineering component and remove the need for user data preprocessing (Yassir et al., 2023). Generally, deep learning methods outperform machine learning. However, they are not always used because they require larger amounts of data for training. For example, fish school classification using machine learning methods requires that the input training data set have already identified schools and additional descriptors to be calculated (Proud et al., 2019). In deep learning, these features would be automatically learned (Yassir et al., 2023). Deep learning methods are deemed unnecessary for simpler cases, such as target spectra classification with limited availability for feature manipulation and training data. However, deep learning would become relevant for broadband species identification on field data with an entire echogram (Roa et al., 2023).
Chapter outline and research objectives
This thesis in structured in 5 chapters. Chapter 1 provides an introduction to the main themes discussed throughout the thesis, Chapter 2, 3 and 4 are research papers and Chapter 6 provides the final conclusions.
In Chapter 2, I conducted a comparative study of the zooplankton and ichthyoplankton density estimates in near-surface sound scattering layers using four different methods. Two of the methods were by direct sampling from a research vessel (mesozooplankton net (MultiNet), macrozooplankton trawl (Tucker trawl)) and the other two methods used broadband acoustic data collected from an autonomous surface vehicle (forward and inverse methods). I compared the density estimates results and contextualized the results in terms of each method's expected biases and uncertainties. I discussed the value of new solutions for surveying ecosystems with emerging technologies.
In Chapter 3, I examined the potential of increasing the taxonomic resolution of acoustic surveys with autonomous platforms by classifying the target spectra of zooplankton. I trained three conceptually different surpervised learning classification algorithms with modelled target spectra of four different Arctic zooplankton groups. I validated the classification predictions against observations collected in a mesocosm of a known mixed zooplankton community. I discussed the limitations of the tested method and provided recommendation for model-infromed classification of zooplankton.
In Chapter 4, we investigated the potential of discriminating between coincident species in the Arctic using only their measured target spectra. we conducted single-species mesocosm experiments to collect the target spectra of free swimming Atlantic cod, polar cod and northern shrimp. The target spectra measurements were used to train three machine learning classification algorithms. we discussed the feasibility of expanding the supervised classification of mesocosm-informed classification for \textit{in situ} measurements of coincident species from a lowered acoustic probe or diving autonomous platform. The publication resulting from this chapter's research was co-led by Chelsey McGowan-Yallop.
In Chapter 5, I summarized the results and contributions of the research from Chapters 2, 3 and 4. I discussed limited availability of ecosystem monitoring data in the Arctic and the use of autonomous platforms equipped with broadband echosounders as a tool to increase the monitoring potential and ecosystem understanding in the Arctic and beyond. I propose directions for future research to incorporate acoustic measurements in ecosystem monitoring.





\section{Chapter outline and research objectives}
This thesis in structured in 5 chapters. Chapter 1 provides an introduction to the main themes discussed throughout the thesis, Chapter 2, 3 and 4 are research papers and Chapter 6 provides the final conclusions.\\
In Chapter 2, I conducted a comparative study of the zooplankton and ichthyoplankton density estimates in near-surface sound scattering layers using four different methods. Two of the methods were by direct sampling from a research vessel (mesozooplankton net (MultiNet), macrozooplankton trawl (Tucker trawl)) and the other two methods used broadband acoustic data collected from an autonomous surface vehicle (forward and inverse methods). I compared the density estimates results and contextualized the results in terms of each method's expected biases and calculated uncertainties. I discussed the importance of new solutions for surveying ecosystems with emerging technologies.\\
In Chapter 3, I examined the potential of increasing the taxonomic resolution of acoustic surveys with autonomous platforms by classifying the target spectra of zooplankton. I trained three conceptually different supervised learning classification algorithms with modelled target spectra of four different Arctic zooplankton groups. I validated the classification predictions against observations collected in a mesocosm of a known mixed zooplankton community. I discussed the limitations of the tested method and provided recommendation for model-informed classification of zooplankton.\\
In Chapter 4, we investigated the potential of discriminating between coincident species in the Arctic using only their measured target spectra. we conducted single-species mesocosm experiments to collect the target spectra of free swimming Atlantic cod, polar cod and northern shrimp. The target spectra measurements were used to train three machine learning classification algorithms. we discussed the feasibility of expanding the supervised classification of mesocosm-informed classification for \textit{in situ} measurements of coincident species from a lowered acoustic probe or diving autonomous platform. The publication resulting from this chapter's research was co-led by Chelsey McGowan-Yallop.\\
In Chapter 6, I summarized the results and contributions of the research from Chapters 2, 3 and 4. I discussed limited availability of ecosystem monitoring data in the Arctic and the use of autonomous platforms equipped with broadband echosounders as a tool to increase the monitoring potential and ecosystem understanding in the Arctic and beyond. I propose directions for future research to incorporate acoustic measurements in ecosystem research.//
